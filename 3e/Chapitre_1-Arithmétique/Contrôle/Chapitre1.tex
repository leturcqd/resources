\documentclass[14 pt]{extarticle}

	\usepackage[frenchb]{babel}
	\usepackage[utf8]{inputenc}  
	\usepackage[T1]{fontenc}
	\usepackage{amssymb}
	\usepackage[mathscr]{euscript}
	\usepackage{stmaryrd}
	\usepackage{amsmath}
	\usepackage{tikz}
	\usepackage[all,cmtip]{xy}
	\usepackage{amsthm}
	\usepackage{varioref}
	\usepackage{geometry}
	\geometry{a4paper}
	\usepackage{lmodern}
	\usepackage{hyperref}
	\usepackage{array}
	 \usepackage{fancyhdr}
\renewcommand{\theenumi}{\alph{enumi})}
	\pagestyle{fancy}
	\theoremstyle{plain}
	\fancyfoot[C]{} 
	\fancyhead[L]{Contrôle}
	\fancyhead[R]{2 octobre 2023}\geometry{
 a4paper,
 total={170mm,257mm},
 left=20mm,
 top=20mm,
 }
	
	
	\title{Chapitre 1-  Arithmétique}
	\date{}
	\begin{document}

\begin{center}{\Large Contrôle Chapitre 1 - Arithmétique}\\ 
 \end{center} 
\subsection*{Exercice 1} 
 
 \begin{enumerate}
 \item Le nombre $1323$ est-il premier ? Justifier votre réponse. 
 \item Le nombre $443$ est-il premier ? 
 \item Trouver tous les diviseurs de $120$. 
 \end{enumerate}
 
\subsection*{Exercice 2} 
\begin{enumerate}
\item Décomposer $260$ et $90$ en facteurs premiers. 
\item Faire la liste des diviseurs de $90$. 
\item Trouver le plus grand diviseur commun de $90$ et $260$.
\end{enumerate} 

 
 \subsection*{Exercice 3 (Inde, 2014)}
 
 Emma et Arthur ont acheté pour leur mariage
 $3 003$ dragées au chocolat et 
 $3 731$ dragées aux amandes. 
 
 \begin{enumerate}
 \item[1.] Arthur propose de répartir ces dragées de
 façon identique dans $20$ corbeilles. 
 Chaque corbeille doit avoir la même composition. Combien lui reste-t-il de dragées non utilisées ? 
 \item[2.] Emma et Arthur changent d'avis et décident de proposer des petits ballotins dont la composition est identique. Ils souhaitent qu'il ne leur reste pas de dragées. \begin{enumerate}
 \item[a.)] Emma propose d'en faire $90$. Ceci convient-il ? Justifier. 
 \item[b.)] Ils se mettent d'accord pour faire un maximum de ballotins. Combien en feront-ils, et quelle sera leur composition ? 
 \end{enumerate}
 \end{enumerate}
 
 \subsection*{Exercice 4}
 
 \begin{enumerate}
 \item Faire la liste des diviseurs de $182$. 
 \item Faire la liste des diviseurs de $1365$.
 \item Simplifier au maximum la fraction $\frac{182}{1365}$. 
 \end{enumerate}
 \newpage
 
\begin{center}{\Large Contrôle Chapitre 1 - Arithmétique}\\ 
 \end{center} 
 \subsection*{Exercice 1} 
 
 \begin{enumerate}
 \item Le nombre $1383$ est-il premier ? Justifier votre réponse. 
 \item Le nombre $449$ est-il premier ? 
 \item Trouver tous les diviseurs de $180$. 
 \end{enumerate}
 
\subsection*{Exercice 2} 
\begin{enumerate}
\item Décomposer $260$ et $90$ en facteurs premiers. 
\item Faire la liste des diviseurs de $90$. 
\item Trouver le plus grand diviseur commun de $90$ et $260$.
\end{enumerate} 

 
 \subsection*{Exercice 3 (Inde, 2014)}
 
 Emma et Arthur ont acheté pour leur mariage
 $3 003$ dragées au chocolat et 
 $3 731$ dragées aux amandes. 
 
 \begin{enumerate}
 \item[1.] Arthur propose de répartir ces dragées de
 façon identique dans $20$ corbeilles. 
 Chaque corbeille doit avoir la même composition. Combien lui reste-t-il de dragées non utilisées ? 
 \item[2.] Emma et Arthur changent d'avis et décident de proposer des petits ballotins dont la composition est identique. Ils souhaitent qu'il ne leur reste pas de dragées. \begin{enumerate}
 \item[a.)] Emma propose d'en faire $90$. Ceci convient-il ? Justifier. 
 \item[b.)] Ils se mettent d'accord pour faire un maximum de ballotins. Combien en feront-ils, et quelle sera leur composition ? 
 \end{enumerate}
 \end{enumerate}
 
 \subsection*{Exercice 4}
 
 \begin{enumerate}
 \item Faire la liste des diviseurs de $273$. 
 \item Faire la liste des diviseurs de $1365$.
 \item Simplifier au maximum la fraction $\frac{182}{1365}$. 
 \end{enumerate}
 
 
 	\end{document}
