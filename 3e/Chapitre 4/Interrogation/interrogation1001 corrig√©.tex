\documentclass[14 pt, fleqn]{extarticle}

	\usepackage[frenchb]{babel}
	\usepackage[utf8]{inputenc}  
	\usepackage[T1]{fontenc}
	\usepackage{amssymb}
	\usepackage[mathscr]{euscript}
	\usepackage{stmaryrd}
	\usepackage{amsmath}
	\usepackage{tikz}
	\usepackage[all,cmtip]{xy}
	\usepackage{amsthm}
	\usepackage{varioref}
	\usepackage{geometry}
	\geometry{a4paper}
	\usepackage{lmodern}
	\usepackage{hyperref}
	\usepackage{array}
	 \usepackage{fancyhdr}
	 \usepackage{float}
\setlength{\mathindent}{1cm}
\renewcommand{\theenumi}{\alph{enumi})}
	\pagestyle{fancy}
	\theoremstyle{plain}
\newcommand{\exo}[8]{
 \ \\ \ \\
 Nom : \ldots\ldots\ldots\ldots\ldots\ldots\ldots\ldots\ldots Prénom : \ldots\ldots\ldots \\ 
 Calculez :\\ \ \\ 
 \[ (- (#5 \times x +   =   \] 
 \[ (-3)^{-#6} = \] 
 \[ #7^2 = \]
 \[ \sqrt{#8} = \]
  }
	\fancyfoot[C]{} 
	\fancyhead[L]{}
	\fancyhead[R]{}\geometry{
 a4paper,
 total={170mm,257mm},
 left=20mm,
 top=20mm,
 }
	
	
	\title{Interrogation chapitre 5}
	\date{}
	\begin{document}
 Nom : \ldots\ldots\ldots\ldots\ldots\ldots\ldots\ldots\ldots Prénom : \ldots\ldots\ldots \\ 
Développez et réduire :
 \begin{eqnarray*} (y+1)\times 2 - (3 - y) \times 2 
 &=& (y\times 2 + 1\times 2) - ( 3\times 2 - y \times 2) \\
 &=& (2 y + 2) - (6 - 2y) \\
 &=& 2y + 2 - 6 + 2y \\
 &=& 4y - 4\end{eqnarray*}
 Convertir : 
 \[ 24,31 \text{m}^3 = 24\ 310\text{dm}^3 =   24\ 310 \text{L}\]
Deux hausses de $10 \%$ sur $100$ euros donnent :$(1+\frac{10}{100})\times (1+\frac{10}{100}) \times 100 = (1,1)^2\times 100 =1,21\times 100= 121$ euros.
 
 \ \\  
 \hrule
 \ \\ \ \\
 Nom : \ldots\ldots\ldots\ldots\ldots\ldots\ldots\ldots\ldots Prénom : \ldots\ldots\ldots \\ 
Développez et réduire :
\begin{eqnarray*}
(x-1)\times 3 - (3 - x) \times 3
&=& ( x \times 3 - 1 \times 3) - ( 3\times 3 - x \times 3)\\
&=& (3x - 3) - (9 -3x) \\
&=& 3x - 3 - 9 + 3x \\
&=& 6x -12 \end{eqnarray*} 
 Convertir : 
 \[ 965,12\text{cm}^3 = 0,965 12\text{dm}^3 = 0,965 12 \text{L}\]
Deux baisses de $10 \%$ sur $100$ euros donnent : 
$(1-\frac{10}{100}) \times (1-\frac{10}{100}) \times 100 = (0,9)^2 \times 100 = 0,81 \times 100 = 81$ euros.
 
 \newpage
 \hrule
 \ \\ \ \\
 Nom : \ldots\ldots\ldots\ldots\ldots\ldots\ldots\ldots\ldots Prénom : \ldots\ldots\ldots \\ 
Développez et réduire :
 \begin{eqnarray*}
(a+2)\times 3 - (2 - a) \times 2
&=& ( a \times 3 +2 \times 3) - (2\times 2 - a \times 2)\\
&=& (3a +6) - (4-2a) \\
&=& 3a + 6 - 4 + 2a \\
&=& 5a +2 \end{eqnarray*} 
 Convertir : 
 \[ 546,124 \text{dm}^3 = 546,124\text{ L} =  546\ 124 \text{mL}\]
 
Deux hausses de $20 \%$ sur $100$ euros donnent :$(1+\frac{20}{100})\times (1+\frac{20}{100}) \times 100 = (1,2)^2\times 100 =1,44\times 100= 144$ euros.
 
 \ \\ 
 \hrule
 \ \\ \ \\
 Nom : \ldots\ldots\ldots\ldots\ldots\ldots\ldots\ldots\ldots Prénom : \ldots\ldots\ldots \\ 
Développez et réduire :
 \begin{eqnarray*}
(a+3)\times 4 - (3 - a) \times 2
&=& ( a \times 4 + 3 \times 4) - (3\times 2 - a \times 2)\\
&=& (4a +12) - (6-2a) \\
&=& 4a + 12 - 6 + 2a \\
&=& 6a +6 \end{eqnarray*} 
 Convertir : 
 \[ 78,124 \text{m}^3 = 78\ 124 \text{dm}^3=  78\ 124 \text{L}\]
Deux baisses de $20 \%$ sur $100$ euros donnent : 
$(1-\frac{20}{100}) \times (1-\frac{20}{100}) \times 100 = (0,8)^2 \times 100 = 0,64 \times 100 = 64$ euros.
 \ \\  
 \hrule
 
 
 
 	\end{document}
