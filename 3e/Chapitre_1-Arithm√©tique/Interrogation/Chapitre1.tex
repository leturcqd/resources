\documentclass[14 pt]{extarticle}

	\usepackage[frenchb]{babel}
	\usepackage[utf8]{inputenc}  
	\usepackage[T1]{fontenc}
	\usepackage{amssymb}
	\usepackage[mathscr]{euscript}
	\usepackage{stmaryrd}
	\usepackage{amsmath}
	\usepackage{tikz}
	\usepackage[all,cmtip]{xy}
	\usepackage{amsthm}
	\usepackage{varioref}
	\usepackage{geometry}
	\geometry{a4paper}
	\usepackage{lmodern}
	\usepackage{hyperref}
	\usepackage{array}
	 \usepackage{fancyhdr}
\renewcommand{\theenumi}{\alph{enumi})}
	\pagestyle{fancy}
	\theoremstyle{plain}
	\fancyfoot[C]{} 
	\fancyhead[L]{Contrôle}
	\fancyhead[R]{ septembre 2023}\geometry{
 a4paper,
 total={170mm,257mm},
 left=20mm,
 top=20mm,
 }
	
	
	\title{Chapitre 1-  Arithmétique}
	\date{}
	\begin{document}

\begin{center}{\Large Contrôle Chapitre 1 - Arithmétique}\\ 
 \end{center}
 
 \subsection*{Exercice 1 (France métropolitaine 2022)}
 
 Une collectionneuse compte ses cartes Pokémon afin de les revendre.
Elle possède 252 cartes de type « feu » et 156 cartes de type « terre »
\begin{itemize}
\item[1.]\begin{enumerate}\item Parmi les trois propositions suivantes, laquelle correspond à la décomposition en produit de facteurs premiers du nombre 252 ? 
\[ 2^2 \times 9 \times 7 \ \ \ \ \ \ \ \ 2 \times 2 \times 3
\times 21 \ \ \ \ \ \ \ \ 2^2\times 3^2\times 7\]
\item Donner la décomposition en produit de facteurs premiers du nombre $156$.
\end{enumerate}
\item[2.] Elle veut réaliser des paquets identiques, c’est-à-dire contenant chacun le même nombre
de cartes « terre » et le même nombre de cartes « feu » en utilisant toutes ses cartes.
\begin{enumerate}
\item Peut-elle faire 36 paquets ?
\item  Quel est le nombre maximum de paquets qu’elle peut réaliser ?
\item Combien de cartes de chaque type contient alors chaque paquet ?
\end{enumerate}


\item[3.] Elle choisit une carte au hasard parmi toutes ses cartes. On suppose les cartes indiscernables au toucher.
Calculer la probabilité que ce soit une carte de type « terre ».
\end{itemize}


 \subsection*{Exercice 2 (Pays du groupe 1 2022)}
 

 
 Pour fêter les 25 ans de sa boutique, un chocolatier souhaite offrir aux premiers clients de la journée une
boîte contenant des truffes au chocolat. Il a confectionné 300 truffes : 125 truffes parfumées au café et 175 truffes enrobées de noix de coco. 

 Il souhaite fabriquer ces boîtes de sorte que :
\begin{itemize}
\item[•]Le nombre de truffes parfumées au café soit le même dans chaque boîte.
\item[•]Le nombre de truffes enrobées de noix de coco soit le même dans chaque boîte.
\item[•]Toutes les truffes soient utilisées.
\end{itemize}

\begin{enumerate}
\item Décomposer 125 et 175 en produit de facteurs premiers.
\item En déduire la liste des diviseurs communs à 125 et 175.
\item Quel nombre maximal de boîtes pourra-t-il réaliser ?
\item Dans ce cas, combien y aura-t-il de truffes de chaque sorte dans chaque boîte ?
\end{enumerate}


 \subsection*{Exercice 3 (Nouvelle-Calédonie 2021)}
 \begin{itemize}
 \item[1.]
 \begin{enumerate}\item Justifier que 330 n'est pas un nombre premier.\\La décomposition en produit de facteurs premiers de 500 est : $500 = 2^2\times 5^3$

 \item Décomposer 330 en produit de facteurs premiers.
 \item Justifier que 165 divise 330.
 \item Justifier que 165 ne divise pas 500.
 \end{enumerate}
 La pâtisserie Délices a préparé $330$ biscuits aux noix et $500$ biscuits au chocolat.\\
\textbf{La pâtisserie souhaite répartir le plus de biscuits possible dans 165 boites.}
 
 \item[2.] Combien de biscuits aux noix y a-t-il dans chaque boîte ?\\ 
 La pâtisserie met aussi le même nombre de biscuits au chocolat dans chaque boîte.
 
 \item[3.]\begin{enumerate}
 \item Combien de biscuits au chocolat y a-t-il dans chaque boîte ?
 \item Combien de biscuits au chocolat reste-t-il ?

 \end{enumerate}
 Une boîte de biscuits coûte 3 650 francs.
À partir de 10 boîtes achetées, la pâtisserie Délices offre une réduction de 5 \% sur le montant total.
 
 \item[4.] 4. Combien va-t-on payer pour l’achat de 12 boîtes?\\ \textbf{Faire apparaître les calculs effectués.}
 \end{itemize}
 
 \subsection*{Exercice 4 (France métropolitaine 2021)}
 
 Un professeur organise une sortie pédagogique au Futuroscope pour ses élèves de troisième. Il veut répartir les 126 garçons et les 90 filles par groupes. Il souhaite que chaque groupe comporte le même nombre de filles et le même nombre de garçons.
 
 \begin{itemize}
 \item[1.] Décomposer en produit de facteurs premiers les nombres 126 et 90
 \item[2.] Trouver tous les entiers qui divisent à la fois les nombres 126 et 90
 \item[3.] En déduire le plus grand nombre de groupes que le professeur pourra
constituer. Combien de filles et de garçons y aura-t-il alors dans chaque groupe ?
 \end{itemize}


\subsection*{Exercice 5 (Nouvelle-Calédonie 2020)}

\begin{itemize}
\item[1.] Justifier que le nombre 102 est divisible par 3.
\item[2.] On donne la décomposition en produits de facteurs premiers de 85 : $85 = 5 \times 17$\\
Décomposer 102 en produit de facteurs premiers.
\item[3.] Donner trois diviseurs non premiers du nombre $102$.\\
Un libraire dispose d'une feuille cartonnée de 85 cm x 102 cm.
Il souhaite découper dans celle-ci, en utilisant toute la feuille, des étiquettes carrées. Les côtés de ces étiquettes ont tous la même mesure.
\item[4.] Les étiquettes peuvent-elles avoir 34 cm de côté ?Justifier.

\item[5.] Le libraire découpe des étiquettes de 17 cm de côté. \\
Combien d'étiquettes pourra-t-il découper dans ce cas ?
\end{itemize}


\subsection*{Exercice 6 (France métropolitaine 2019)}

Le capitaine d'un navire possède un trésor constitué de 69 diamants, 1 150 perles et 4 140 pièces d'or.

\begin{itemize}
\item[1.] Décomposer 69 ; 1 150 et 4 140 en produits de facteurs premiers.
\item[2.] Le capitaine partage équitablement le trésor entre les marins.\\
Combien y-a-t-il de marins sachant que toutes les pièces, perles et diamants ont été distribués ?
\end{itemize}
 	\end{document}
