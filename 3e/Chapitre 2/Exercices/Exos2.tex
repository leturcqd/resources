\documentclass[14 pt, fleqn]{extarticle}

	\usepackage[frenchb]{babel}
	\usepackage[utf8]{inputenc}  
	\usepackage[T1]{fontenc}
	\usepackage{amssymb}
	\usepackage[mathscr]{euscript}
	\usepackage{stmaryrd}
	\usepackage{amsmath}
	\usepackage{tikz}
	\usepackage[all,cmtip]{xy}
	\usepackage{amsthm}
	\usepackage{varioref}
	\usepackage{geometry}
	\geometry{a4paper}
	\usepackage{lmodern}
	\usepackage{hyperref}
	\usepackage{array}
	 \usepackage{fancyhdr}
	 \usepackage{float}\usepackage{setspace}
\setlength{\mathindent}{1cm}
\renewcommand{\theenumi}{\alph{enumi})}
	\pagestyle{fancy}
	\theoremstyle{plain}
	\fancyfoot[C]{} 
	\fancyhead[L]{}
	\fancyhead[R]{}\geometry{
 a4paper,
 total={170mm,257mm},
 left=5mm,
 top=5mm,
 bottom = 0mm
 }
	
	
	\title{Exercices de calcul}
	\date{}
	\begin{document}
	 
 \subsection*{Jour 1}
 
 \begin{enumerate}
 \item Calculez $\frac23+ \frac34$, $\frac23 - \frac34$, puis $\frac23\times\frac34$, et $\frac23\div \frac34$ .
 \item Calculez et réduisez $2 - (x-3)$, puis $3- (2x + 2)$, et $2\times (5+x)+ x$.
 \item Exprimez le volume d'un pavé droit dont les côtés mesurent $8$ cm, $5$ cm et $4$ cm. Donnez une valeur en litres. 
 \end{enumerate}
 \subsection*{Jour 2}
 
 \begin{enumerate}
 \item Calculez $\frac23+ \frac34\times \frac59$, $\frac23\div\frac14 - \frac34$.
 \item Calculez et réduisez $2(x+2) + 3(x-3)$, puis $3+ 2(2x + 2)$, et $2\times (5+x) - (x-1)$.
 \item Exprimez le volume d'un cylindre de rayon $3$ cm et de hauteur $5$ cm. Donnez une valeur exacte, et une valeur approchée au mL.
 \end{enumerate}
 
 \subsection*{Jour 3}
 
 \begin{enumerate}
 \item Calculez $(-\frac14) + (-\frac32) + \frac52$ et $(-7) - (-9) - 17 + (-41) + 1$.
 \item Calculez et réduisez $2(x+2) - (x-3)$ et $3- 2(2x + 2)$.
 \item Exprimez le volume d'un cône de rayon $3$ cm et de hauteur $5$ cm. Donnez une valeur exacte, et une valeur approchée au mL.
 \end{enumerate}
 \subsection*{Jour 4}
 
 \begin{enumerate}
 \item Calculez $(5-3+7-1) + (-9+4-1) - (-3-7+2)$ et $\left( -\frac23 - \frac35 + 1\right) - \left(\frac13+\frac45 - 2\right) + \left (-1+ \frac73\right)$.
 \item Supprimez correctement les parenthèses et crochets dans $[12-(14 - 5 + 1) ] + [-14 + (3-2)]$ et dans $[(5-9)+(3-5)] - [(7+3-5)-(7-10)]$. 
 \item Calculez $2$h$15$min + $11$h$47$min$-3$h$17$min.
 \end{enumerate}
 
 
 \subsection*{Jour 5}
 
 \begin{enumerate}
 \item Calculez $(\frac45)^3 - (\frac23)^{-2}$
 \item Supprimez correctement les parenthèses dans 
 $(a-b+c) - (d-e-f) + (b-a)$, dans
 $( (a-b) - (a-5)) + (b-7 - (a-3) )$
 et dans $(12-(a-b)+6)-(15+(b-a-13) )$.
 \item On choisit un nombre $x$, on lui ajoute $4$, on multiplie le résultat par $2$, et on lui retire le nombre de départ. Traduisez ce programme par une expression littérale. 
 \item Développez et réduisez $2 + x + 4(1-x)$, puis $4 - x + x(1-2x)$ et $5(1+x^3) - x^2(3 - x)$.
 \end{enumerate}
 \newpage
  \subsection*{Jour 6}
 
 \begin{enumerate}
 \item Réduisez 
 $$ - \frac32x+ \frac54x - 3x^2 + \frac{x}6 - \frac52 x^2 + 5 + 4 x^2,$$
 $$ \frac32 x^2 + xy + y^2 - 2yx + \frac{x^2}3 - \frac32x^2,$$
 et $$ \frac25 a^2b + 3a^3 - 4ab^2 + \frac52 a^2b + \frac72 b^3 - b^3 + 2 b^2a.$$ 
 \item On choisit deux nombres $x$ et $y$, on calcule la somme et la différence de ces nombres, puis on forme le produit de ces deux nombres . Traduisez ce programme par une expression littérale. 
 \item On regarde les expressions littérales $-4x^3 - 2x +2$ et $4x-6x^2+5x^3 - 2$. 
 Calculez leur somme, puis leur différence. 
 Calculez la somme et la différence des expressions obtenues, puis divisez les par $2$. Que remarque-t-on ? 
 \end{enumerate}
 
 \subsection*{Jour 7}
 On considère les programmes suivants : 
 
 Programme A : «
 on prend deux nombres ; on calcule leur somme et leur différence ; on calcule ensuite la moitié de la somme de ces deux derniers nombres. »
 
 Programme B :« on prend deux nombres ; on calcule leur somme et leur différence ; on calcule ensuite la moitié de la différence entre ces deux derniers nombres. »
 \begin{enumerate}
 \item Suivez les deux programmes avec les nombres $0$ et $10$. Qu'observe-t-on ? 
 \item Suivez les deux programmes avec les nombres $-3$ et $4$.  Qu'observe-t-on ? 
 \item Suivez les deux programmes avec les nombres $-\frac23$ et $\frac37$.  Qu'observe-t-on ? 
 \item On note désormais $x$ et $y$ les deux nombres choisis. Exprimez en fonction des nombres $x$ et $y$ le nombre obtenu à la fin du programme $A$. Qu'avez-vous montré ainsi ? 
 \item Exprimez en fonction des nombres $x$ et $y$ le nombre obtenu à la fin du programme $B$. Qu'avez-vous montré ainsi ? 
 \end{enumerate}
 	\end{document}
