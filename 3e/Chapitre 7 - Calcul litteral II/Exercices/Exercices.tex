\documentclass[12 pt]{extarticle}

	\usepackage[frenchb]{babel}
	\usepackage[utf8]{inputenc}  
	\usepackage[T1]{fontenc}
	\usepackage{amssymb}
	\usepackage[mathscr]{euscript}
	\usepackage{stmaryrd}
	\usepackage{amsmath}
	\usepackage{tikz}
	\usepackage[all,cmtip]{xy}
	\usepackage{amsthm}
	\usepackage{varioref}
	\usepackage{geometry}
	\geometry{a4paper}
	\usepackage{lmodern}
	\usepackage{hyperref}
	\usepackage{array}
	 \usepackage{fancyhdr}
	 \usepackage{float}
	\pagestyle{fancy}
	\theoremstyle{plain}
	\fancyfoot[C]{\thepage} 
	\fancyhead[L]{Fiche d'exercices}
	\fancyhead[R]{2022-2023}
	
	
	\title{Exercices Chapitre 7}
	\date{}
	\begin{document}

\begin{center}{\Large Exercices de calcul littéral\footnote{Tirés du manuel de 4e Lebossé, Hémery, possibles fautes de recopie.}}\\
 \end{center} 
 
 \subsection*{Exercice 1 : Expressions littérales}
\begin{enumerate}
\item Calculez la valeur numérique des expressions suivantes : 
\begin{enumerate}
\item $3a^2-2a+5$ pour $a =  +4$, $-2$, et $-1$.
\item $3x^2-5x+7$ pour $x =  +3$, $+5$, et $-3$.
\item $4x^3-12x^2-4x+7$ pour $a =  +5$, $-3$, et $\frac12$.
\item $\frac{7x^2}2- \frac{8x-6}3 + \frac{3x+7}4$ pour $a =  +3$, $-2$, et $+5$.
\item $\frac{7-3x}{12}+\frac{2(x-2)}3 +\frac54$ pour $a =  +4$, $+2$, et $-3$.
\item $(x^2+1)^2 - x^4 -2x^2$ pour $a =  -1$, $+\frac12$, et $-\frac12$.
\item $(a^2+1)(a^2-1) + 4a^2$ pour $a =  -3$, $+3$, et $+\frac13$.
\item $\frac{(x+y)^2-(x^2+y^2)}{xy}$ pour $x=-5$ et $y=+2$.
\item $\frac{a^2-ab}{a^2-2ab+b^2}$ pour $a=+7$ et $b=-2$.
\end{enumerate}
\item Vérifiez que les expressions suivantes donnent la même valeur pour $a=-5$ et $b=3$ : 
\[ a^2 - ab + b^2; \ \ \ \ \ \frac{a^3+b^3}{a+b}; 
\ \ \ \ \ (a+b)^2-3ab\]
\item Vérifiez que les expressions suivantes donnent la même valeur pour $a=+4$ et $b=-1$ : 
\[ (a+b)^2(a-b); \ \ \ \ \ (a^2-b^2)(a+b); 
\ \ \ \ \ \frac{a^4+b^4-2a^2b^2}{a-b}\]
\item Peut-on calculer pour $a=+5$, et $b=-2$ la valeur de l'expression : \[ \frac{a^2+2ab-3b^2}{a(a-1)-5b^2}.\]
\item Peut-on calculer pour $a=+1$, et $b=+2$ la valeur de l'expression : \[ \frac{3a^2+5b^2}{4a^2-b^2}.\]
\item Réduisez les expressions suivantes et calculer leurs valeurs numériques :
\begin{enumerate}
\item $\left(-\frac23\right) a^2x \times (-3y) \times \left(+\frac25\right)$ pour $a=-3$, $x=2$ et $y=-1$. 
\item $xy \times \left(-\frac23\right)x^2\times \frac34a^2$ pour $a=+5$, $x=-2$, et $y=+3$.
\item $\frac27a^2\times\left(-\frac34\right)xy^3\times \left(-\frac25\right)a^2x$ pour $a=+3,5$, $x=+3$ et $y=-2$. 
\item $\left(-\frac35\right)a^2 \times \left(\frac23\right)b^2x \times (-x^4)$ pour $a=4$, $b=-1$, et $x=-2$. 
\item $4x^3 \times (-3y^2) \times \left(-\frac56\right) a^2x^2y^5$ pour $a=-\frac12$, $x=+4$, et $y=\frac32$. 
\end{enumerate}
\item Effectuez les sommes suivantes : 
\begin{enumerate}
\item $\frac23ax-\frac12ax+\frac34ax-\frac56ax$.
\item $-\frac35a^2bx+\frac14a^2bx-\frac72a^2bx+\frac1{10}a^2bx$.
\item $-\frac47a^2b^3x+\frac52a^2b^3x-\frac54a^2b^3x$.
\item $\frac34a^2b^3x^4y-\frac23a^2b^3x^4y+\frac14a^2b^3x^4y$. 
\end{enumerate}
\end{enumerate}


\subsection*{Exercice 2 - Réduire une somme}

\begin{enumerate}
\item Réduisez et ordonnez les expressions suivantes: 
\begin{enumerate}
\item $-\frac32 x + \frac54 x-3x^2+\frac{x}6-\frac52 x^2+5+4x^2$.
\item $\frac32 x^2+xy+y^2-2xy+\frac{x^2}3-\frac32x^2$.
\item $4a^2-\frac23a-\frac35a^2+\frac13a-5a-\frac2{15}a^2$.
\item $3x^2+\frac45 - \frac53x - 2x^2 -\frac35x^3 + 4 - 2 x^2 + 7x$. 
\item $4x^2 - \frac72 + \frac35x - \frac52x^2 + \frac43 x^3 - 5 + \frac32 x^3 + 7 - 2x$. 
\item $\frac25 a^2b + 3a^3 -4ab^2 + \frac52 a^2b + \frac72b^3 - b^3 + 2ab^2$. 
\end{enumerate}
\item On pose $A= -4x^3 - 2x+2$ et $B= 4x-6x^2 + 5x^3 - 2$. \begin{enumerate}
\item Calculez $A+B$. 
\item Calculez en $x=2$ les valeurs numériques de $A$, $B$ et $A+B$ pour vérifier la réponse précédente. 
\item Calculez $A-B$.
\item Calculez en $x=3$ les valeurs numériques de $A$, $B$ et $A-B$ pour vérifier la réponse précédente. 
\end{enumerate}
\item Réduisez les expressions suivantes : 
\begin{enumerate}
\item$ \left(-5 x^4 + 3 - \frac45 x^3\right) + \left( 
-\frac23 x^3 - 2x\right) - \left( 7x^2 - \frac45 x + 5x^4\right)$.
\item $(12x^3+2x^2-5x+13) + (3x+5-4x^3)- (5x^3-8+2x^2)$.
\item $(a^3-3a^2b+3ab^2-b^3) + (a^3+3a^2b+3ab^2+b^3) - (6ab^2-3a^3)$. 
\end{enumerate}
\item Effectuez : \begin{enumerate}
\item $(3x-5)+[2x-5 - (3x-2y+4) - (4x-3y-9)]$.
\item $(2x-5y+7) - [(3x+2y-3)-(4x+4y-2)]-[2x-(3y+4)]$. 
\item $[(x-2y+5) - (3x+2y+7)]-[(2x+3)-(4y-2)]$. 
\end{enumerate}
\item Posons : $A=3x^2-4x+5$, $B=2x^2+5x-4$, $C= 4x^2-x+3$. 
Calculez et réduisez : $A+B+C$, $A+B-C$, $A-B+C$, et $-A+B+C$. 
\item Posons : $A=5a^2-3ab+7b^2$, $B=6a^2-8ab+9b^2$, $C= 4a^2-3ab-7b^2$. 
Calculez et réduisez : $A-B-C$, $-A-B+C$ et $-A+B-C$.
\item Posons : $P=2x^5-3x^2+4x$, $Q=4x^3 - 5x^2 +2x-1$, $R = 4x^5 - 2x^3 + 3x -1$, et $S= 3x^2+2x-5$. 
Calculez et réduisez : $(P+Q)-(R+S)$, $(P-Q)+(R-S)$, et $P-Q-R+S$. 
\end{enumerate}

\subsection*{Exercice 3 - Développer et réduire un produit}

\begin{enumerate}
\item Effectuez les produits suivants : 
\begin{enumerate}
\item $(3a^2b^3)\left(\frac23ab^5\right)$.
\item $\left(\frac45a^3b^2c\right)\left(-\frac34abc^4\right)$. 
\item $\left(\frac47a^2xy^3\right)\left(-\frac52a^3y^4\right)$.
\item $\left(-\frac34x^2y\right)\left(+\frac35a^3y^5\right)$.
\item $\left(\frac94a^4x^2y^3\right)\left(-\frac43ax^2\right)$. 
\item $\left(\frac{14}3a^2b^3x\right)\left(-\frac67a^2b^5\right)$. 
\item $\left(-\frac72ax^2y\right)\left(-\frac8{15}b^3xy^2\right)\left(\frac5{21}abx^3\right)$.
\item $\left(-\frac23xy^2\right)^2(-4x^2y)$.
\item $\left(\frac5{12}a^4b^2x\right)\left(-\frac27ax^2y^3\right)\left(-\frac{14}5b^2xy^4\right)$.
\item $\left(\frac35x^2y\right)^3\left(-\frac54xy\right)$.
\end{enumerate}
\item Calculez : \begin{enumerate}
\item $\left(-\frac25ab^3\right)^2$.
\item $\left(\frac53a^2b^3x^4\right)^2$.
\item $\left(-\frac32a^4b^3y^2\right)^3$.
\item $\left(\frac72a^3b^5x^3\right)^2$.
\item $\left(-\frac94a^4b^2x^5\right)^2$.
\item $\left(-\frac65ax^4y^5\right)^3$.
\end{enumerate}
\item Effectuez les produits suivants : 
\begin{enumerate}
\item $\left(\frac32a^2b-\frac54ab+3a\right)\left(-\frac43a^2b^3\right)$.
\item $\left(\frac54ax^2+\frac32bx-4x\right)\left(-\frac45ax^5\right)$.
\item $\left(\frac25a^2x-3ay-4by\right)(4a^3x^2y)$.
\item $\left(-\frac32x^5+\frac{15}4x^3-\frac25x\right)\left(-\frac{20}3x^4\right)$.
\item $(2x-3y)(4x-2)$.
\item $(2a+3b)(-4a+6b)$.
\item $(-4x+3y+1)(y-3)$. 
\item $(-2a+3b-5)(a-b)$. 
\item $(2x^3-3y-2+5)(x^2-y)$.
\item $(4a^3-5b^4+ab)(a^2-b)$.
\item $(5xy+3x-2y)(2x-y)$.
\item $(-3xy+4x-2y)(x+5)$.
\item $(14a^2b+5a^2-b)(a^2-2b)$.
\item $(7a^3b-4b^2+2a^3)(2a^3+4b^2)$. 
\end{enumerate}
\item Soient les expressions littérales $A=-2x^2+3x+5$ et $B=x^2-x+3$. 
\begin{enumerate}
\item Calculez le produit $AB$. 
\item Pour $x=-3$, vérifiez le résultat en calculant séparément les valeurs numériques de $A$, $B$ et $AB$. 
\end{enumerate}
\item Soit l'expression littérale $A= x^2-3x+2$. \begin{enumerate}
\item Calculez le carré, puis le cube de $A$. 
\item Vérifiez pour $x=-4$, les valeurs de $A$, $A^2$ et $A^3$. 
\end{enumerate}
\item Effectuez les produits suivants.
\begin{enumerate}
\item $(2x-7)(-3x+2)$.
\item $(4x^5+7-2x^3)(x^3-2x)$.
\item $(5x^3-2x)(3x-4x^2)$
\item $(2x-7x^2+5x^3)(3x-5x^2+8)$.
\item $\left(-2x+\frac32\right)(4x+3)$.
\item $\left(\frac83x-\frac32x^2+5\right)(4x^3-5x^2+7)$.
\item $(7x^4-2x^3+4x^2)(3x^2-5)$.
\item $(2x^2-4x^3)(x^3-2x)$.
\item $(2x^2-4+2x)(x^2+5-2x)$.
\item $\left(\frac54x^3-2x+\frac12\right)\left(\frac72x^3-\frac23x+x^2\right)$.
\end{enumerate}
\item Calculez les produits suivants : 
\begin{enumerate}
\item $(2x+3)(3x+2)(x-4)$.
\item $(5x-1)(2x+3)(7+4x)$.
\item $(3x^2-1)(x+1)(x-1)$. 
\item $\left(x-\frac35\right)(5x^2-1)(5x+3)$.
\item $(2x^2+3x-4)^2$. 
\item $(4x^3-7x+2x^2+5)^2$. 
\item $(7x-5)^3$. 
\item $(x^2-x+2)^3$
\end{enumerate}
\item Développez et réduisez : 
\begin{enumerate}
\item $5(3a^2-4b^3)-[9(2a^2-b^3)-2(a^2-5b^3)]$.
\item $3a^2(2b-1)-[2a^2(5b-3)-2b(3a^2+1)]$.
\item $(2a+5b)(3a-2b)-(2a-1)(3a+2b)-(a-2b)(5b-1)$.
\item $(2x-3y)(5x-2y)-(3x-2y)(2x+1)-(5x-y)(3y+1)$.
\item $(ax^2-b)(ax^2-2b)+3b(ax^2-b)+b(b-1)$.
\item $(x-1)(x-2)(x-3)+6(x-1)(x-2)+7(x-1)$.
\item $(x^2+y^2)(x^2-y^2)(x-y)+xy(x^3+y^3)$.
\item $\frac23x^2y\left(2x^2-\frac{y}3\right)-2x^2(2x^2-1)+\left(2x^2-\frac{y}3\right)\left(1-\frac{y}3\right)(2x^2-1)$;
\end{enumerate}
\end{enumerate}

\subsection*{Exercice 4 - Identités remarquables}

\begin{enumerate}
\item Vérifiez les identités suivantes : 
\begin{enumerate}
\item $\frac12(a+b)^2+\frac12(a-b)^2=a^2+b^2$.(identité du parallélogramme).
\item $\left(\frac{a+b}2\right)^2-\left(\frac{a-b}2\right)^2=ab$. (identité de polarisation). 
\item $(a-b)(a^3+a^2b+ab^2+b^3)=a^4-b^4$.
\item $(a+b)(a^3-a^2b+ab^2-b^3)=a^4-b^4$.
\item $(x^2+x+1)(x^2-x+1)= x^4+x^2+1$.
\item $(aa'+bb')^2+(ab'-a'b)^2=(a^2+b^2)(a'^2+b'^2)$.
\item $(x-1)(x+1)(x^2+1)=(x-1)(x^3+x^2+x+1)=x^4-1$.
\item $a(b-c)+b(c-a)+c(a-b)=0$.
\item $a(bz-cy)+b(cx-az)+c(ay-bx)=0$.
\item $(x+y)^3-3xy(x+y)=x^3+y^3$.
\item $(x+y)^3+2(x^3+y^3)=3(x+y)(x^2+y^2)$. 
\end{enumerate}
\item Utilisez les identités remarquables du cours pour développer les produits suivants : 
\begin{enumerate}
\item $\left(\frac32x^3-\frac25y^2\right)^2$.
\item $\left(\frac43x^5+\frac25y^3\right)^2$.
\item $\left(\frac25x^2-\frac34y\right)\left(\frac25x^2+\frac34y\right)$
\item $\left(\frac23a^2x^3-\frac12by^4\right)\left(\frac23a^2x^3+\frac12by^4\right)$.
\item $(3x+4y-5)(3x+4y+5)$.
\item $\left(\frac23x-\frac45y-1\right)\left(\frac23x+\frac45y+1\right)$.
\item $(3x+4y-2z)^2$. 
\item $\left(\frac52x-\frac34y+z\right)^2$. 
\end{enumerate}
\item Développez et réduisez : 
\begin{enumerate}
\item $(a+b)(a+x)(b+x)-a(b+x)^2-b(a+x)^2$.
\item $bc(b-c)+ca(c-a)+ab(a-b)+(b-c)(c-a)(a-b)$.
\item $(a+b+c)[(a-b)^2+(b-c)^2+(c-a)^2]$.
\item $(b-c)(x-a)^2+(c-a)(x-b)^2+(a-b)(x-c)^2$.
\item $(a+b)^2+(b+c)^2+(c+a)^2-(a+b+c)^2$. 
\item $a^2(a-b)(a-c)+b^2(b-c)(b-a)+c^2(c-a)(c-b)$. 
\end{enumerate}
\end{enumerate}

\subsection*{Exercice 5- Équations}

\begin{enumerate}
\item Résolvez les équations suivantes : 
\begin{enumerate}
\item $5(2x-3) - 4(5x-7)=19-2(x+11)$. 
\item $4(x+3)-7x+17 = 8(5x-1)+166$. 
\item $17-14(x+1)= 13-4(x+1)-5(x-3)$. 
\item $5x+3,5+(3x-4) = 7x-3(x-0,5)$. 
\item $7(4x+3)-4(x-1)=15(x+0,75)+7$. 
\item $17x+15(x-1)=-1-14(3x+1)$. 
\end{enumerate}
\item Résolvez les équations suivantes (après développement, les termes en $x^2$ ou $x^3$ se simplifient) : 
\begin{enumerate}
\item $(x-1)^2+(x+3)^2= 2(x-2)(x+1)+38$.
\item $5(x^2-2x-1)+2(3x-2)=5(x+1)^2$.
\item $(9x+1)(x-2)=(3x+4)(3x-5)$.
\item $7(3-2x)-5x(2x-1)=(5x+3)(3-2x)$. 
\item $(3x-1)^2-(2x+3)^2+7 = (2x+1)(2x-1) + x(x+7)$.
\item $(x+2)^3+(x-2)^3+(x+1)^3=3(x+1)(x+2)(x-2)$.
\end{enumerate}
\item Résolvez les équations suivantes : 
\begin{enumerate}
\item $\frac52x+3 - \frac{7x}4 = x + \frac94$. 
\item $\frac{3x}7 - \frac{2x}{15} + 3 = \frac{x}3+\frac{13}3$.
\item $x+\frac12-\frac{x}6 = 16 - \frac{2x}9 + \frac13$. 
\item $\frac{7x}4-2-\frac{x}2=\frac{2x}{13}-\frac{85}{52}$. 
\item $\frac{2x}3+4 -\frac{2x}5 = \frac{x}2-\frac{x}3+3,5$. 
\item $\frac{x}6-1=\frac{x}4-\frac{x}3-1$. 
\end{enumerate}
\item Résolvez les équations suivantes (on se ramènera au cas entier en multipliant par un nombre opportun).
\begin{enumerate}
\item $\frac{x+5}4 - \frac{x-3}6 = \frac{x}3$. 
\item $\frac{3x-7}2 + \frac{x+1}3 = -16$. 
\item $x-\frac{x+1}3 = \frac{2x+1}5$. 
\item $\frac{7-3x}12 + \frac34 = 2(x-2) + \frac{5(5-2x)}6$. 
\item $\frac{x}5 - \frac{3x-1}6 + \frac{3-x}4 = 0$.
\item $\frac{3(x+3)}4 + \frac12 = \frac{5x+9}3 - \frac{7x-9}4$. 
\item $\frac{2x-7}5 + \frac{x+11}2 = -4$. 
\item $\frac{2x-3}3 - \frac{x-3}6 = \frac{4x+3}4 - 17$. 
\item $\frac{5x-3}4 - \frac{7x-5}9 = \frac{x+19}6$. 
\item $\frac{5x+1}8-\frac{x-1}3 = \frac{4(2x-3)}9$. 
\item $\frac{2x-1}3 - \frac{5x+2}7 = x +13$. 
\item $\frac{8x+2}5 - \frac{x-11}7 = \frac{5x-3}2 - \frac{3x-1}4$.
\item $\frac{2x-7}9 - \frac{x-5}6 = \frac{x-9}8$. 
\item $\frac{5x+7}4 - \frac{3x+5}8 = \frac{4x+9}5 - \frac{x-9}3$. 
\item $\frac{5x+6}7 - \frac{3x+1}4 = \frac{x+16}5$. 
\item $\frac{4x+7}5 -\frac{x-5}6 = \frac{2x+14}3 - \frac{2x-7}9$. 
\end{enumerate}
\item Résoudre les équations suivantes (après multiplication et développement, les termes en $x^2$ disparaissent) : 
\begin{enumerate}
\item $\frac{(x-1)(x+5)}3 - \frac{(x+2)(x+5)}12 = \frac{(x-1)(x+2)}4$.
\item $\frac{(x+1)^2}3 + \frac{(x-2)(x-3)}2 = \frac{(5x-1)(x-4)}6 + \frac{28}3$. 
\item $\frac{(3x+1)(3x-1)}9 - \frac{(x-5)(x+1)}2 = \frac{(9x-1)(x+3)}{18} + \frac89$. 
\item $\frac{(4x+7)^2}4 - \frac{(5x-1)^2}7 = \frac{(8x-3)(3x+4)-79x}{56}$. 
\item $\left(x-\frac83\right)(x+0,75) = (x+4,5)(x+1,5) - \frac{145}3$. 
\item $\frac{(x-5)^2}5 + \frac{(x+3)^2}3 = \frac{(3x+1)(3x-1)-x(x+1)}{15}$. 
\item $\left(3x-\frac45\right)\left(5x+\frac23\right)
=15 (x-1)(x+1)+\frac7{15}$. 
\end{enumerate}
\end{enumerate}

\subsection*{Exercice 6 - Équations produits}
\begin{enumerate}
\item Résoudre les équations produits suivantes :
\begin{enumerate}
\item $(x-1)(x+2)(x-3)=0$. 
\item $(x-3)(x-4)(x-5)=0$. 
\item $(2x+1)(x+1)(4x-3) = 0$.
\item $(2x+1)(x+4)(3x+1)=0$. 
\item $x(5x+1)(4x-3)(3x-4)=0$. 
\item $5x(3x-7)=0$. 
\end{enumerate}
\item Résoudre en factorisant pour faire apparaître une équation produit : 
\begin{enumerate}
\item $x^2-3x=0$. 
\item $5x^2+8x=0$. 
\item $4x^2-\frac{7x}3=0$. 
\item $\frac{x^2}5+x=0$.
\item $-\frac{3x^2}5 + x =0$. 
\item $-\frac{5x^2}7 - \frac{3x}4=0$. 
\item $x(x+1)=x+1$.
\item $(4x-1)(x-3)=(x-3)(5x+2)$.
\item $(x+3)(x-5)+(x+3)(3x-4)=0$.
\item $5(x+1)(x+2)(x-3)=4(x+1)(x+2)(x-4)$. 
\end{enumerate}
\item Résoudre les équations suivantes, en utilisant des identités ou des factorisations pour se ramener à une équation produit : 
\begin{enumerate}
\item $(x+5)(4x-1)+x^2-25=0$.
\item $(x+4)(5x+9)-x^2+16=0$.
\item $x^2-9=0$.
\item $5x^2-125=0$.
\item $4x^2-49=0$. 
\item $x^2-100=0$.
\item $x^2=81$.
\item $9x^2=64$. 
\item $(x+1)^2-(2x-5)^2=0$.
\item $(2x+7)^2-(4x-9)^2=0$.
\item $(5x+1)^2=(x-1)^2$.
\item $(3x+1)^2=(x-4)^2$.
\item $4(x+1)^2-9(x-1)^2=0$.
\item $(x+7)^2-81(x-5)^2=0$.
\item $5x^3-5x=0$.
\item $(x+1)(x-1)^2-(x+1)(x-2)^2=0$.
\item $3x^2-12x=0$. 
\item $(3x+1)(x-3)^2=(3x+1)(2x-5)^2$. 
\item $7x^3-175x=0$. 
\item $(x+5)(3x+2)^2=x^2(x+5)$. 

\end{enumerate}
\end{enumerate}
 	\end{document}
