\documentclass[14 pt]{extarticle}

	\usepackage[frenchb]{babel}
	\usepackage[utf8]{inputenc}  
	\usepackage[T1]{fontenc}
	\usepackage{amssymb}
	\usepackage[mathscr]{euscript}
	\usepackage{stmaryrd}
	\usepackage{amsmath}
	\usepackage{tikz}
	\usepackage[all,cmtip]{xy}
	\usepackage{amsthm}
	\usepackage{varioref}
	\usepackage{geometry}
	\geometry{a4paper}
	\usepackage{lmodern}
	\usepackage{hyperref}
	\usepackage{array}
	 \usepackage{fancyhdr}
\renewcommand{\theenumi}{\alph{enumi})}
	\pagestyle{fancy}
	\theoremstyle{plain}
	\fancyfoot[C]{} 
	\fancyhead[L]{Interrogation chapitre 2}
	\fancyhead[R]{6 novembre 2023}\geometry{
 a4paper,
 total={170mm,257mm},
 left=20mm,
 top=20mm,
 }
	
	
	\title{Chapitre 2 -  Calcul littéral I}
	\date{}
	\begin{document}

\begin{center}{\Large Interrogation de calcul}\\ 
 \end{center}


\subsection*{Exercice 1 - Vrai ou faux (3 points)}

Pour chacune des affirmations suivantes, dire si elle est vraie ou non. 

\begin{enumerate}
\item $\frac{1}{2} + \frac{2}{3} = \frac{1+2}{2+3}$

\item $\frac{1}{2} \times \frac{4}{5} = \frac{1\times4}{2\times5}$

\item $\frac{1}{3} \times\frac{2}{3} = \frac{1\times 2}{3}$

\end{enumerate}

\subsection*{Exercice 2 - Développements (7 points)}

Développer et réduire les expressions suivantes : 

\begin{enumerate}
\item $2\times (3 + x) $
\item $(5 + x - y) \times 4$
\item $-(x+y-2)$
\item $(2 + x) \times 2 - (3+ x) \times 4$

\end{enumerate}


\subsection*{Exercice 3 - Factorisations (6 points)}

Factoriser et écrire sous la forme la plus simple possible : 
\begin{enumerate}
\item $2\times 97 + 8 \times 97$ 
\item $3\times x + x \times 2$ 
\item $ 40 a - 20 b + 10$
\item $ ab + 2a + 5a$
\end{enumerate}



\subsection*{Exercice 4 (4 points)}

Effectuer les calculs suivants : 

\begin{enumerate}
\item $\frac25 \times (\frac13-\frac1{21})$
\item $\frac34 \div (\frac1{15}+\frac1{24})$
\item $\frac3{28} - \frac{9}{70}$ 
\end{enumerate}

\newpage  

\begin{center}{\Large Interrogation de calcul}\\ 
 \end{center}




\subsection*{Exercice 1 - Vrai ou faux (3 points)}

Pour chacune des affirmations suivantes, dire si elle est vraie ou non. 

\begin{enumerate}
\item $\frac{1}{2} \times \frac{8}{5} = \frac{1\times8}{2\times5}$
\item $\frac{1}{2} + \frac{7}{4} = \frac{1+7}{2+4}$

\item $\frac{1}{7} \times\frac{2}{7} = \frac{1\times 2}{7}$

\end{enumerate}

\subsection*{Exercice 2 - Développements (7 points)}

Développer et réduire les expressions suivantes : 

\begin{enumerate}
\item $2\times (5 + y) $
\item $(x + 5  - y) \times 2$
\item $-(x - y + 2)$
\item $(3 + x) \times 2 - (2+ x) \times 4$

\end{enumerate}


\subsection*{Exercice 3 - Factorisations (6 points)}

Factoriser et écrire sous la forme la plus simple possible : 
\begin{enumerate}
\item $97 \times 71 + 3 \times 71$ 
\item $3\times y + y \times 7$ 
\item $ 50 a - 20 b + 10$
\item $ ab + 3a -5a$
\end{enumerate}



\subsection*{Exercice 4 (4 points)}

Effectuer les calculs suivants : 

\begin{enumerate}
\item $\frac15 \times (\frac13-\frac1{21})$
\item $\frac14 \div (\frac1{15}+\frac1{24})$
\item $\frac3{28} - \frac{9}{70}$ 
\end{enumerate}


 	\end{document}
