\documentclass[14 pt]{extarticle}

	\usepackage[frenchb]{babel}
	\usepackage[utf8]{inputenc}  
	\usepackage[T1]{fontenc}
	\usepackage{amssymb}
	\usepackage[mathscr]{euscript}
	\usepackage{stmaryrd}
	\usepackage{amsmath}
	\usepackage{tikz}
	\usepackage[all,cmtip]{xy}
	\usepackage{amsthm}
	\usepackage{varioref}
	\usepackage{geometry}
	\geometry{a4paper}
	\usepackage{lmodern}
	\usepackage{hyperref}
	\usepackage{array}
	 \usepackage{fancyhdr}
	 \usepackage{float}
\renewcommand{\theenumi}{\alph{enumi})}
	\pagestyle{fancy}
	\theoremstyle{plain}
	\fancyfoot[C]{} 
	\fancyhead[L]{Droites du triangle}
	\fancyhead[R]{2023-2024}\geometry{
 a4paper,
 total={170mm,257mm},
 left=20mm,
 top=20mm,
 }
	
	
	\title{Exercices chapitre 4}
	\date{}
	\begin{document}

\begin{center}{\Large Exercices chapitre 4}\\ 
 \end{center}
 \subsection*{Exercice 1}
  
 Est-il possible de construire un triangle avec : 
 
 \begin{enumerate}
 \item des longueurs de $3$, $4$ et $5$ centimètres.
 \item des longueurs de $20$, $2$ et $15$ centimètres
 \item des longueurs de $2$ décimètres, $3$ centimètres, et $1$ décimètre
 \item des longueurs de $15$ millimètres, $2$ centimètres et $0,03$ mètres.
 \end{enumerate}
 Si oui, construisez un tel triangle, sinon, justifiez votre réponse. 


 \subsection*{Exercice 2}
 
 
On considère un quadrilatère $ABCD$. Montrez qu'on a \[ AB - BC - CD < AD < AB + BC + CD\] 


\subsection*{Exercice 3 - Tracé de l'hexagone régulier}

\begin{enumerate}

\item Construisez un triangle équilatéral $ABC$ de côté $8$ cm. 
\item Construisez les médiatrices des côtés de $ABC$.
\item Placez leur point de concours $O$, ainsi que le cercle $(\mathcal C)$ passant par $A$, $B$ et $C$. 
\item Placez $A'$ le point d'intersection de la médiatrice de $[BC]$ avec $(\mathcal C)$, $B'$ le point d'intersection de la médiatrice de $[AC]$ avec $(\mathcal C)$, et $C'$ le point d'intersection de la médiatrice de $[AC]$ avec $(\mathcal C)$.
\item Tracez le polygone $AB'CA'BC'$. 
\end{enumerate}

\subsection*{Exercice 4 - Tracé de l'octogone régulier}


\begin{enumerate}

\item Construisez un carré $ABCD$ de côté $8$ cm. 
\item En vous inspirant de l'exercice précédent, construisez des points $E$, $F$, $G$ et $H$ de telle sorte que $AEBFCGDH$ soit un octogone régulier.
\end{enumerate}


\subsection*{Exercice 5 - Tracé du pentagone régulier à la règle et au compas}


\begin{enumerate}

\item Construisez un carré $ABCD$ de côté $8$ cm. 
\item Tracez en pointillés la médiatrice de $[CD]$, qui coupe ce segment en son milieu $E$. 
\item Placez le centre $O$ du carré, à l'intersection de la médiatrice précédente et d'une des diagonales du carré. 
\item Tracez le cercle $(\mathcal C)$ de centre $O$ passant par $E$. 
\item Prolongez en pointillés la demi-droite $[DC)$, puis utilisez le compas depuis le point $E$ pour placer le point de $[DC)$ tel que $ET=EB$. 
\item Tracez en pointillés la médiatrice de $[DT]$ afin de placer le milieu $I$ de $[DT]$. 
\item Tracez en pointillés le triangle $EOH$ isocèle en $H$ tel que $EH= DI$. 
\item Le côté $[OH]$ rencontre le cercle $(\mathcal C)$. Placez le point d'intersection $M$.
\item Trouvez le point $N$ de $(\mathcal C)$ tel que $EM= MN$. 
\item Trouvez le point $P$ de $(\mathcal C)$ tel que $MN=NP$. 
\item Trouvez le point $Q$ de $(\mathcal C)$ tel que $NP=PQ$. 
\item Tracez en rouge le pentagone $EMNPQ$. 
\end{enumerate}

\subsection*{Exercice 6}
Soit un triangle $ABC$ avec $AB=3$ cm, $AC = 4$ cm, et $BC= 5$ cm. On admet que ce triangle est rectangle en $A$. 
\begin{enumerate}
\item Tracez le triangle $ABC$. 
\item Tracez la hauteur issue de $A$, et placez son pied $H$. 
\item En considérant les aires, démontrez que $AB \times AC = AH \times BC$. 
\item En déduire la valeur de la longueur $AH$, sans la mesurer. 
\end{enumerate}

\subsection*{Exercice 7}

On veut construire un triangle $ABC$ d'aire $10$ cm${}^2$ avec $AB= 5$ cm, et $AC= 6$ cm.

\begin{enumerate}
\item Tracez le segment $[AB]$.
\item Quel est l'ensemble des points $C$ tels que $AC=6$ cm ?
\item Si on note $H$ le point de $[AB]$ tel que $(AB)$ et $(CH)$ soient perpendiculaires, exprimez l'aire du triangle $ABC$ en fonction de $AB$ et $HC$. 
\item Concluez-en que $CH= 4$ cm. 
\item Construisez l'ensemble des points situés à $4$ cm de $(AB)$. 
\item Déduisez-en comment construire le triangle $ABC$. 
\end{enumerate} 





 

 	\end{document}
