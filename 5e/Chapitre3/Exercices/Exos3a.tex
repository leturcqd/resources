\documentclass[14 pt]{extarticle}

	\usepackage[frenchb]{babel}
	\usepackage[utf8]{inputenc}  
	\usepackage[T1]{fontenc}
	\usepackage{amssymb}
	\usepackage[mathscr]{euscript}
	\usepackage{stmaryrd}
	\usepackage{amsmath}
	\usepackage{tikz}
	\usepackage[all,cmtip]{xy}
	\usepackage{amsthm}
	\usepackage{varioref}
	\usepackage{geometry}
	\geometry{a4paper}
	\usepackage{lmodern}
	\usepackage{hyperref}
	\usepackage{array}
	 \usepackage{fancyhdr}
	 \usepackage{float}
\renewcommand{\theenumi}{\alph{enumi})}
	\pagestyle{fancy}
	\theoremstyle{plain}
	\fancyfoot[C]{} 
	\fancyhead[L]{Feuille d'exercices}
	\fancyhead[R]{ novembre 2022}\geometry{
 a4paper,
 total={170mm,257mm},
 left=20mm,
 top=20mm,
 }
	
	
	\title{Exercices chapitre 3 - 1}
	\date{}
	\begin{document}

\begin{center}{\Large Exercices chapitre 3 - 1}\\ 
 \end{center}
 \subsection*{Exercice 1}
 

\begin{enumerate}
\item Écrire la fraction dont le numérateur est $4$ et le dénominateur est $7$.
\item Écrire une fraction ayant $12$ comme numérateur. 
\item Écrire une fraction ayant $3$ comme dénominateur. 
\item Écrire une fraction égale à $3$ ayant $9$ comme numérateur. 
\item Écrire une fraction égale à $4$ ayant $7$ comme dénominateur.
\end{enumerate}

\subsection*{Exercice 2 }
Pour chacune des fractions suivantes, 
rechercher son écriture décimale. Si celle-ci est finie, la donner en entier. Sinon, soulignez le motif qui se répète. 

\begin{enumerate}
\item $\frac45$
\item $\frac1{25}$ 
\item $\frac{2}{7}$
\item $\frac3{64}$ 
\item $\frac{3}{11}$
\item $\frac4{125}$ 
\item $\frac{4}{9}$
\end{enumerate}
\subsection*{Exercice 3}

Compléter les pointillés. 

\begin{enumerate}
\item $\frac34= \frac{3 \times \ldots}{4 \times \ldots}= 
\frac{\ldots}{16}$
\item $\frac2{11}= \frac{2 \times \ldots}{11 \times \ldots}= 
\frac{\ldots}{99}$
\item $\frac5{12}=  \frac{5 \times \ldots}{12 \times \ldots}= 
\frac{\ldots}{16}$
\item $\frac{13}{11}=  \frac{13 \times \ldots}{11 \times \ldots}= 
\frac{\ldots}{143}$
\end{enumerate}

Simplifier les fractions suivantes (on cherchera toujours la forme
irréductible). 
\begin{enumerate}
\item $\frac{12}{15}$. 
\item $\frac{24}{18}$
\item $\frac{21}{49}$ 
\item $\frac{12}{32}$
\item $\frac{81}{243}$
\item $\frac{105}{315}$
\item $\frac{125}{325}$
\item $\frac{143}{169}$
\item $\frac{531}{351}$
\end{enumerate}


\subsection*{Exercice 4}

Regroupez les fractions suivantes par fractions égales. 

\[ \frac43, \ \  
 \frac{9}{15}, \ \    
 \frac{24}{28}, \ \  
 \frac{18}{30}, \ \  
  \frac{48}{36}, \ \  
   \frac67, \ \  
      \frac{24}{18}, \ \  
       \frac{3}{5}\]


 	\end{document}
