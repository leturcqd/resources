\documentclass[14 pt]{extarticle}

	\usepackage[frenchb]{babel}
	\usepackage[utf8]{inputenc}  
	\usepackage[T1]{fontenc}
	\usepackage{amssymb}
	\usepackage[mathscr]{euscript}
	\usepackage{stmaryrd}
	\usepackage{amsmath}
	\usepackage{tikz}
	\usepackage[all,cmtip]{xy}
	\usepackage{amsthm}
	\usepackage{varioref}
	\usepackage{geometry}
	\geometry{a4paper}
	\usepackage{lmodern}
	\usepackage{hyperref}
	\usepackage{array}
	 \usepackage{fancyhdr}
\renewcommand{\theenumi}{\alph{enumi})}
	\pagestyle{fancy}
	\theoremstyle{plain}
	\fancyfoot[C]{} 
	\fancyhead[L]{Contrôle}
	\fancyhead[R]{19 septembre 2022}\geometry{
 a4paper,
 total={170mm,257mm},
 left=20mm,
 top=20mm,
 }
	
	
	\title{Contrôle Chapitre 1}
	\date{}
	\begin{document}

\begin{center}{\Large Contrôle Chapitre 1}\\ 
 \end{center}
 
 
 \subsection*{Exercice 1}
Écrire en langage mathématique les expressions suivantes.
\begin{enumerate}
\item La différence entre $14$ et $2$.
\item La somme de $2$ et du quotient de $6$ par $3$.
\item Le produit de $3$ par la somme de $4$ et $7$. 
\end{enumerate}

\subsection*{Exercice 2}

Décrire en français les expressions suivantes. 
\begin{enumerate}
\item $(15 \div 3) - 4$
\item $3 + (4\times 5)$.
\end{enumerate}

\subsection*{Exercice 3}

Effectuer les calculs suivants en numérotant les opérations et en détaillant les étapes. \begin{enumerate}
\item $10 + 3 \times 4$
\item $ 7 \times 3 + 2$
\item $18\div 3 \times 3$
\item $(2+5) \times (3+2) - 2$
\item $ (26 - 1 + 5)\div 2 \times 5$
\end{enumerate}

\subsection*{Exercice 4}

Parmi les égalités suivantes, certaines sont fausses, et d'autres vraies. Recopier chaque égalité en rajoutant des parenthèses si nécessaire pour les rendre toutes vraies.

\begin{enumerate}
\item $18 \div 2 \div 3 = 3$
\item $27 - 1 + 3 = 23$
\item $(4+ 6 )\times 5-3 = 20$
\item $4 \times 9\times 7 - 13 = 200$. 
\end{enumerate}

\subsection*{Exercice 5}
Factoriser les expressions suivantes, puis effectuer le calcul. 

\begin{enumerate}
\item $2\times 9 + 2\times 91$
\item $24 \times 3 + 3 \times 26$
\item $12\times 112 - 3\times 4 \times 12$
\end{enumerate}

\subsection*{Exercice 6}

On achète trois stylos à $2$ euros l'unité, et cinq cahiers à $4$ euros l'unité. Écrire le prix total sous la forme d'un seul calcul. A-t-on besoin de parenthèses ? 

\subsection*{Exercice 7}

Un rectangle fait $123,4$ cm de longueur et $2,01$ cm de largeur. 
\begin{enumerate}
\item Écrire son périmètre sous la forme d'un seul calcul. \textbf{On ne demande pas d'effectuer ce calcul ensuite.}
\item Si le résultat de la question précédente présentait des parenthèses, sont-elles nécessaires ? 
\item Si les parenthèses étaient nécessaires à la question précédente, trouver un calcul sans parenthèses donnant le même résultat. 
\end{enumerate}

 	\end{document}
