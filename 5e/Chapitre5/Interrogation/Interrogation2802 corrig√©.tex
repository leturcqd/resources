	\documentclass[14pt]{extreport}
\usepackage{extsizes}
	\usepackage[frenchb]{babel}
	\usepackage[utf8]{inputenc}  
	\usepackage[T1]{fontenc}
	\usepackage{amssymb}
	\usepackage[mathscr]{euscript}
	\usepackage{stmaryrd}
	\usepackage{amsmath}
	\usepackage{tikz}
	\usepackage[all,cmtip]{xy}
	\usepackage{amsthm}
	\usepackage{varioref}
	\usepackage[ margin=1in]{geometry}
	\geometry{a4paper}
	\usepackage{lmodern}
	\usepackage{hyperref}
	\usepackage{array}
	\usepackage{float}
	\usepackage{easytable}
	 \usepackage{fancyhdr}\usepackage{longtable}
	 \usetikzlibrary{shapes.misc}
\newlength{\taillecellule}
\setlength{\taillecellule}{2cm}
\newcolumntype{C}{@{}>{\centering\arraybackslash}p{\taillecellule}@{}}

\usepackage{pstricks,multido}
\usepackage{arrayjob}
\usepackage{calc,xlop}
\tikzset{cross/.style={cross out, draw=black, minimum size=2*(#1-\pgflinewidth), inner sep=0pt, outer sep=0pt},
%default radius will be 1pt. 
cross/.default={1pt}}

	\pagestyle{fancy}
	\theoremstyle{plain}
	\fancyfoot[C]{\empty} 
	\fancyhead[L]{Interrogation chapitre 5}
	\fancyhead[R]{27 février 2024}
	
	
	\title{Interrogation chapitre 5}
	\date{}
	\begin{document}



\subsection*{Exercice 1}  % 4 points

Recopiez et remplissez les pointillés :

\[ a)\ 8 + (-3) = 5 \ \ \ \ \ \ \ \ \ \ \ \ \ \ 
 b)\ 9 - (-5) = 9 + 5 = 14 \]
\[ c)\ -1 + 5 = 4  \ \ \ \ \ \ \ \ \ \ \ \ \ \  
d)\ -1 - (-8) = 7\]

\subsection*{Exercice 2} % 1 / 1.5 / 1.5  -> 4 points

Calculez les sommes suivantes en détaillant vos étapes. 

\[ -1  + 3 + 2 + (-4) + (-5) + 6  = 11- 10 = 1\]
\[ (-1  + 3 + 4) - ((-4) + (-7) + 6 ) =  (7-1) - (6-11) = 6 - (-5) = 6+5 = 11\]
\[ -1  - ( 3 + 2 + (-4) ) - ((-5) + 6)= -1 - 1 - 1 = -3 \]
\[ 1,03  + 3,4 + 6 + (-4,13)   + (-5,4) + 2,1 = 12,53 - 9,53=3 \]

\subsection*{Exercice 3}

Remplissez les pyramides additives suivantes (chaque nombre est la somme des deux nombres en-dessous de lui.)

\begin{figure}[H]
\center
\begin{tikzpicture}[scale=.9,every node/.style={draw,minimum width=1.8cm,minimum height=.9cm}]
\draw (0,0)  node {{9}};
\draw(-1,-1) node { {8}} ++(2,0) node { {1}};
\draw(-2,-2) node {{6,5}} ++(2,0) node {{1,5}} ++(2,0) node {{-0,5}};
\draw(-3,-3) node {{}{7,5}} ++(2,0) node {{}{-1}} ++(2,0) node {{}{2,5}} ++(2,0) node {{}{-3}};
\end{tikzpicture}
\
\begin{tikzpicture}[scale=.9,every node/.style={draw,minimum width=1.8cm,minimum height=.9cm}]
\draw (0,0)  node {{2,7}};
\draw(-1,-1) node { {1,2}} ++(2,0) node { {}{1,5}};
\draw(-2,-2) node {{1,8}} ++(2,0) node {{}{-0,6}} ++(2,0) node {{2,1}};
\draw(-3,-3) node {{}{4,1}} ++(2,0) node {{}{-2,3}} ++(2,0) node {{1,7}} ++(2,0) node {{0,4}};
\end{tikzpicture}
\end{figure}


\subsection*{Exercice 4} % 6 points
Complétez le carré magique suivant avec des entiers relatifs pour que les quatre lignes, les quatre colonnes, 
et les deux diagonales aient une somme égale à $6$. 
\[
\begin{TAB}(e,1cm){|c|c|c|c|}{|c|c|c|c|}
    1 & 4 &-1 &2  \\
    1 & 7 & -3 & 1 \\
     -1 & 2 & 0 & 5 \\
    5  & -7 & 10 & -2
\end{TAB}
\]


 
\newpage
\subsection*{Exercice 1}  % 4 points

Recopiez et remplissez les pointillés :

\[ a)\ 7 + (-3) =4 \ \ \ \ \ \ \ \ \ \ \ \ \ \ 
 b)\ 9 - (-4) = 13 \]
\[ c)\ -3 + 7 = 4  \ \ \ \ \ \ \ \ \ \ \ \ \ \  
d)\ -1 - (-4) = 3\]
\subsection*{Exercice 2} % 1 / 1.5 / 1.5  -> 4 points

Calculez les sommes suivantes en détaillant vos étapes. 

\[ -1  + 4 + 2 + (-3) + (-5) + 6 = (4+2+6)-(1+3+5) = 12 - 9 = 3\]
\[ (-1  + 5 + 4) - ((-2) + (-7) + 6 ) = (9-1) -(6-9) =8 - (-3) = 11 \]
\[ -1  - ( 6 + 2 + (-4) ) - ((-5) + 6) = -1 - 4 - 1 = -6\]
\[ 1,03  + 3,3 + 7 + (-4,03)   + (-5,4) + 2,1 =13,43- 9,43 = 4  \]



\subsection*{Exercice 3}

Remplissez les pyramides additives suivantes (chaque nombre est la somme des deux nombres en-dessous de lui.)

\begin{figure}[H]
\center
\begin{tikzpicture}[scale=.9,every node/.style={draw,minimum width=1.8cm,minimum height=.9cm}]
\draw (0,0)  node {{9}};
\draw(-1,-1) node { {8}} ++(2,0) node { {1}};
\draw(-2,-2) node {{6,5}} ++(2,0) node {{1,5}} ++(2,0) node {{-0,5}};
\draw(-3,-3) node {{}{7,5}} ++(2,0) node {{}{-1}} ++(2,0) node {{}{2,5}} ++(2,0) node {{}{-3}};
\end{tikzpicture}
\ 
\begin{tikzpicture}[scale=.9,every node/.style={draw,minimum width=1.8cm,minimum height=.9cm}]
\draw (0,0)  node {{0}};
\draw(-1,-1) node {{} {-5,9}} ++(2,0) node { {5,9}};
\draw(-2,-2) node {{-5,7}} ++(2,0) node {{}{-0,2}} ++(2,0) node {{6,1}};
\draw(-3,-3) node {{}{2,4}} ++(2,0) node {{-8,1}} ++(2,0) node {{7,9}} ++(2,0) node {{}{-1,8}};
\end{tikzpicture}
\end{figure}

\subsection*{Exercice 4} % 6 points

Complétez le carré magique suivant avec des entiers relatifs pour que les quatre lignes, les quatre colonnes, 
et les deux diagonales aient une somme égale à $6$. 
%\[
%\begin{TAB}(e,1cm){|c|c|c|c|}{|c|c|c|c|}
%    1 & & & 2 \\
%    1 &  & -3 &  \\
%      & 2 & 0 & \\
%      & -7 &  & -2
%\end{TAB}
%\]
%
%%
\[
\begin{TAB}(e,1cm){|c|c|c|c|}{|c|c|c|c|}
    -2 &10 & -7& 5 \\
    5 & 0 & 2 & -1 \\
     1 & -3 & 7 &1 \\
     2 & -1 & 4 & 1
\end{TAB}
\]
 
\end{document}