\documentclass[14 pt]{extarticle}

	\usepackage[frenchb]{babel}
	\usepackage[utf8]{inputenc}  
	\usepackage[T1]{fontenc}
	\usepackage{amssymb}
	\usepackage[mathscr]{euscript}
	\usepackage{stmaryrd}
	\usepackage{amsmath}
	\usepackage{tikz}
	\usepackage[all,cmtip]{xy}
	\usepackage{amsthm}
	\usepackage{varioref}
	\usepackage{geometry}
	\geometry{a4paper}
	\usepackage{lmodern}
	\usepackage{hyperref}
	\usepackage{array}
	 \usepackage{fancyhdr}
	 \usepackage{float}
\renewcommand{\theenumi}{\alph{enumi})}
	\pagestyle{fancy}
	\theoremstyle{plain}
\newcommand{\exo}{
 \ \\ \ \\
 Nom : \ldots\ldots\ldots\ldots\ldots\ldots\ldots\ldots\ldots Prénom : \ldots\ldots\ldots \\ 
 Calculez :
  }
	\fancyfoot[C]{} 
	\fancyhead[L]{}
	\fancyhead[R]{}\geometry{
 a4paper,
 total={170mm,257mm},
 left=20mm,
 top=20mm,
 }
	
	
	\title{Interrogation chapitre 5}
	\date{}
	\begin{document}
 
 \exo
 \[ (4 + (-3) ) \times 5 =  \]
 \[ 2 \times 3 + 1 = \]
 \[ 3 \div 3 + 1 = \]
 \hrule
 \exo
 
 \[ (7 + (-3) ) \times 5 =  \]
 \[ 2 \times 4 + 1 = \]
 \[ 6  \div 2 + 1 = \]
 \hrule\exo 
 
 \[ (4 + (-2) ) \times 5 =  \]
 \[ 3 \times 3 + 1 = \]
 \[ 12 \div 3 + 1 = \]
 \hrule\exo 
 
 \[ (6 + (-3) ) \times 5 =  \]
 \[ 4 \times 3 + 1 = \]
 \[ 12 \div 2 + 1  =\]
 
 
 	\end{document}
