	\documentclass[14 pt]{extarticle}

	\usepackage[frenchb]{babel}
	\usepackage[utf8]{inputenc}  
	\usepackage[T1]{fontenc}
	\usepackage{amssymb}
	\usepackage[mathscr]{euscript}
	\usepackage{stmaryrd}
	\usepackage{amsmath}
	\usepackage{tikz}
	\usepackage[all,cmtip]{xy}
	\usepackage{amsthm}
	\usepackage{varioref}
	\usepackage{geometry}
	\geometry{a4paper}
	\usepackage{lmodern}
	\usepackage{hyperref}
	\usepackage{array}
	 \usepackage{fancyhdr}

	\pagestyle{fancy}
	\theoremstyle{plain}
	\fancyfoot[C]{\thepage} 
	\fancyhead[L]{Fiche d'exercices}
	\fancyhead[R]{2022-2023}
	\newcounter{n}
	\numberwithin{n}{section}
	\newtheorem{df}[n]{Définition}
	\newtheorem{theo}{Théorème}
	\labelformat{theo}{théorème}
    \newtheorem{rmq}[n]{Remarque}
	\labelformat{rmq}{remarque~#1}
	\newtheorem{cor}[n]{Corollaire}
	\newtheorem{lm}{Lemme}
	\labelformat{lm}{lemme~#1}
	\newtheorem{hyp}[n]{Hypothèse}
	\newtheorem{nt}[n]{Notation}

	\renewcommand\epsilon{\varepsilon}
	\renewcommand\phi{\varphi}
	\newcommand\R{\mathbb{R}}
	\newcommand\s{\mathbb{S}}
	
	
	
	\title{Exercices Chapitre 3-C}
	\date{}
	\begin{document}

\begin{center}{\Large  Exercices chapitre 3 (feuille 3)}\\ 
 \end{center}


\textbf{Exercice 1 - Écritures décimales}

Pour les fractions suivantes, calculer leur écriture décimale, vérifier si elle est finie ou non, et, dans le second cas, trouver la période qui se répète. 
\[ \frac{1}{11}, \ \ \frac{3}{15}, \ \ \frac{3}{7}, \ \ \frac{5}{35}, \ \ \frac{7}{125}.\]

\textbf{Exercice 2 - Simplifications de fractions}

Simplifier (sans calculatrice) les fractions suivantes :
 $$\frac{15}{85}, \frac{84}{60},
\frac{105}{135}, \frac{105}{147},
\frac{1 815}{2 385}, \frac{2 184}{7 560}, 
\frac{3 339}{5 880}, \frac{17 226}{18 810}$$

\textbf{Exercice 3 - Fractions égales}
\begin{itemize}
\item Trouver une fraction égale à $\frac{23}{5}$ dont le dénominateur est $105$. 
\item Trouver une fraction égale à $\frac{4}{19}$ dont le numérateur est $32$. 
\end{itemize}


\textbf{Exercice 4 - Fractions égales (avancé)}
\begin{itemize}
\item Trouver une fraction égale à $\frac{11}{17}$ dont la somme des termes (numérateur et dénominateur) est égale à $56$. 
\item Trouver une fraction égale à $\frac{23}{5}$ dont la somme des termes est égale à $112$. 
\item Trouver une fraction égale à $\frac{4}{19}$ dont la différence du dénominateur et du numérateur est $60$. 
\item Trouver une fraction égale à $\frac{52}{117}$ dont la somme des termes est égale à $325$. (Indication.\footnote{Simplifier la fraction avant toute chose.})
\end{itemize}

\textbf{Exercice 5 - Réduction au même dénominateur}

Pour chacun des couples de fractions suivants, les réduire au même dénominateur, les comparer, et les additionner. 
\[ \frac{3}{5}\text{ et } \frac{7}{10}\]
\[ \frac{3}{8}\text{ et } \frac{7}{12}\]
\[ \frac{4}{7}\text{ et } \frac{7}{4}\]
\[ \frac{3}{5}\text{ et } \frac{4}{7}\]
\[ \frac{3}{15}\text{ et } \frac{4}{14}\]
\newpage
\textbf{Exercice 6 - Repérage.}

Tracer une demi-droite graduée de 24 carreaux en plaçant 1 au douzième carreau, et y placer les fractions suivantes : 

\[  \frac{3}{36}, \ \ \frac{15}{12}, \ \ \frac{7}{3}, \ \ \frac{5}{6}, \ \ \frac{121}{66}   \]


\textbf{Exercice 7 - Écriture fractionnaire d'un nombre décimal (I)}

1) Calculer l'écriture décimale de $\frac19$.

2) En déduire que $0,444444\ldots$ s'écrit comme une fraction (d'entiers). 

\textbf{Exercice 8 - Écriture fractionnaire d'un nombre décimal (II)}

1) Calculer l'écriture décimale de la fraction $\frac1{99}$. 

2) En déduire que $0,010101\ldots$ s'écrit comme une fraction. 

3) Écrire $0,2323232323\ldots$ comme le produit d'un entier et de $0,01010101\ldots$. En déduire que 
$0,2323232323\ldots$ s'écrit comme une fraction. 


\textbf{Exercice 9 - Écriture fractionnaire d'un nombre décimal (III)}

1) Calculer l'écriture décimale de $\frac1{999}$. 

2) En déduire que $0,321321321321\ldots$ s'écrit comme une fraction. La simplifier. 

\emph{Remarque} : En utilisant la même méthode qu'aux trois exercices précédents, on peut toujours obtenir, 
en partant d'un nombre dont le développement décimal est infini mais périodique une fraction d'entiers égale à ce nombre. De plus, le dénominateur de cette fraction peut toujours être pris plus petit (et même divisant) $9...9$, où le nombre de $9$ est le nombre de chiffres de la période qui se répète. En général, cela peut se simplifier : par exemple $0,3333\ldots = \frac39 = \frac13$. 
	\end{document}
