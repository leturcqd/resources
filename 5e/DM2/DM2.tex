\documentclass[12 pt]{extarticle}

	\usepackage[frenchb]{babel}
	\usepackage[utf8]{inputenc}  
	\usepackage[T1]{fontenc}
	\usepackage{amssymb}
	\usepackage[mathscr]{euscript}
	\usepackage{stmaryrd}
	\usepackage{amsmath}
	\usepackage{tikz}
	\usepackage[all,cmtip]{xy}
	\usepackage{amsthm}
	\usepackage{varioref}
	\usepackage{geometry}
	\geometry{a4paper}
	\usepackage{lmodern}
	\usepackage{hyperref}
	\usepackage{array}
	 \usepackage{fancyhdr}
\renewcommand{\theenumi}{\alph{enumi})}
	\pagestyle{fancy}
	\theoremstyle{plain}
	\fancyfoot[C]{} 
	\fancyhead[L]{Devoir maison}
	\fancyhead[R]{Mai 2024}\geometry{
 a4paper,
 total={170mm,257mm},
 left=20mm,
 top=20mm,
 }
	
	
	\title{Devoir maison}
	\date{}
	\begin{document}

\begin{center}{\Large Devoir maison}\\ 
 \end{center}
 
 \subsection*{Exercice 1 }
 
 
 Effectuer les opérations suivantes : 
 \begin{enumerate}
 \item  $ 7 - 10 + 14 -13 - 21 +3$.
 \item $ 0,75 +4,7-0,7 -0,5 -1$. 
 \item $(5-3+7-1)+(-9+4-1)-(-3-7+2)$. 
 \item $[(5-9)+(3-5)]-[(7+3-5)-(7-10)]$. 
 \item $[12-(14-5+0,75)] + [-15 + (3,25-2)]$. 
 
  \end{enumerate}
 \subsection*{Exercice 2}
 
 Calculez les durées entre les moments suivants : 
 
\begin{enumerate}
\item 2h 15 min et 11h45 min. 
\item -7h14 min et +8h 16 min. 
\item -3h10 min et -1h 5min.
\item -4h17min51s et +12h17min 47s. 
\end{enumerate}

\subsection*{Exercice 3 - Parallélogrammes}
 
 On considère un quadrilatère ABCD. On dit que ABCD est un \emph{parallélogramme} si les côtés opposés sont parallèles entre eux, c'est-à-dire $(AB)//(CD)$ et $(BC)//(AD)$.
 
 \begin{enumerate}
\item[1.] Dans un premier temps, on suppose que ABCD est un parallélogramme. \begin{enumerate}
 \item Faire une figure d'un parallélogramme $ABCD$ en 
 prolongeant tous les côtés en droites. 
 \item Faire apparaître sur la figure un angle alterne-interne à $\widehat{BAD}$. 
 \item En déduire que la somme de deux angles consécutifs du parallélogramme est $180^o$. 
 \end{enumerate}
 \item[2.] Supposons désormais que $\widehat{BAD}+\widehat{ADC}=180^o$, que $\widehat{ADC}+\widehat{DCB}=180^o$, que $\widehat{DCB}+\widehat{CBA}=180^o$ et que $\widehat{CBA}+\widehat{BAD}=180^o$. 
 \begin{enumerate}
 \item Trouver un angle alterne-interne à $\widehat{BAD}$ de sommet $D$ sur la figure. 
 \item Exprimer cet angle en fonction de $\widehat{ADC}$.
 \item Montrer qu'il est donc égal à $\widehat{BAD}$. 
 \item En utilisant une propriété du cours, montrer que $(CD)//(AB)$. 
 \item En adaptant ce qui précède, montrer également que $(AD)//(BC)$. 
 \end{enumerate}
 \end{enumerate}
 
 On a donc montré la propriété suivante : « Un 
 quadrilatère est un parallélogramme exactement si ses angles consécutifs sont supplémentaire (de somme $180^o)$. »
 \begin{enumerate}
 \item[3.] À partir de ce qui précède, montrer qu'un quadrilatère est un parallélogramme exactement si ses angles opposés sont égaux. 
 \end{enumerate}
 


 	\end{document}
