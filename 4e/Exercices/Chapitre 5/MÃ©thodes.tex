\documentclass[14 pt, twoside]{extarticle}

	\usepackage[frenchb]{babel}
	\usepackage[utf8]{inputenc}  
	\usepackage[T1]{fontenc}
	\renewcommand{\baselinestretch}{1.25} 
	\usepackage{amssymb}
	\usepackage[mathscr]{euscript}
	\usepackage{stmaryrd}
	\usepackage{amsmath}
	\usepackage{tikz}
	\usepackage[all,cmtip]{xy}
	\usepackage{amsthm}
	\usepackage{varioref}
	\usepackage[a4paper, marginparwidth= 4 cm, total={14cm, 25 cm}]{geometry}
	\geometry{a4paper}
	\usepackage{lmodern}
	\usepackage{hyperref}
	\usepackage{array}
	 \usepackage{fancyhdr}
	 \usepackage{float}
	\pagestyle{fancy}
	\theoremstyle{plain}
	\fancyfoot[C]{\thepage} 
	\fancyhead[L]{Modèles}
	\fancyhead[R]{2022-2023}
	
\renewcommand{\theenumi}{\alph{enumi})}

\newlength{\taillecellule}
\setlength{\taillecellule}{2cm}
\newcolumntype{C}{@{}>{\centering\arraybackslash}p{\taillecellule}@{}}

\usepackage{pstricks,multido}
\usepackage{arrayjob}
\usepackage{calc,xlop}

	\title{Modèles de rédaction chapitre 5}
	\date{}
	\begin{document}
	\maketitle
	
	
	\section{Théorème de Thalès : calculer une longueur}


Dans le triangle $ABC$, on sait que : \marginpar{\color{red}On commence\\ par fixer \\le cadre.}

\begin{itemize}
\item $M\in [AB]$, 
\item $N\in [AC]$, 
\item $(MN)//(BC)$. 

$\underbrace{\text{D'après le théorème de Thalès}}_{\text{\color{red}on rappelle le théorème utilisé}}$, on a les égalités : $\underbrace{\frac{AM}{AB} = \frac{AN}{AC} = \frac{MN}{BC}}_{\text{\color{red}on conclut}.}$. 

{\color{red}On détermine ensuite les longueurs manquantes à partir de l'égalité trouvée, en faisant des règles de trois.}
\end{itemize}

	
		
	\section{Contraposée du théorème de Thalès : Montrer un non-parallélisme}
	
	Dans le triangle $ABC$, on sait que : \marginpar{\color{red}On commence\\ par fixer \\le cadre.}

\begin{itemize}
\item $M\in [AB]$, 
\item $N\in [AC]$, 
\item $\displaystyle \frac{AM}{AB}\neq \frac{AN}{AC}$. En effet, 
$\displaystyle\frac{AM}{AB} = \frac\cdots\cdots$ et $\displaystyle\frac{AN}{AC} = \frac\cdots\cdots$, \\et $\cdots \times \cdots \neq \cdots \times \cdots$. {\color{red} On a calculé les produits en croix.}

$\underbrace{\text{D'après la contraposée du théorème de Thalès}}_{\text{\color{red}on rappelle le théorème utilisé}}$,\\
 $\underbrace{\text{les droites $(MN)$ et $(BC)$ ne sont pas parallèles.}}_{\text{\color{red}on conclut}.}$ 

\end{itemize}
\newpage
	
	\section{Réciproque du théorème de Thalès : Montrer un parallélisme}
	
	
	
	
 	\end{document}
