\documentclass[12 pt]{extarticle}

	\usepackage[frenchb]{babel}
	\usepackage[utf8]{inputenc}  
	\usepackage[T1]{fontenc}
	\usepackage{amssymb}
	\usepackage[mathscr]{euscript}
	\usepackage{stmaryrd}
	\usepackage{amsmath}
	\usepackage{tikz}
	\usepackage[all,cmtip]{xy}
	\usepackage{amsthm}
	\usepackage{varioref}
	\usepackage{geometry}
	\geometry{a4paper}
	\usepackage{lmodern}
	\usepackage{hyperref}
	\usepackage{array}
	 \usepackage{fancyhdr}
	 \usepackage{float}
	\pagestyle{fancy}
	\theoremstyle{plain}
	\fancyfoot[C]{\thepage} 
	\fancyhead[L]{Fiche d'exercices}
	\fancyhead[R]{2022-2023}
	
	
	\title{Exercices Chapitre 1}
	\date{}
	\begin{document}

\begin{center}{\Large Chapitre 1 - Proportionnalité}\\
 \end{center} 

\subsection*{Exercice 1}

Parmi les situations suivantes, reconnaître lesquelles sont des situations de proportionnalité : \begin{itemize}
\item[a)] Le poids d'un sac de pommes de terre et leur poids.
\item[b)] Le prix d'un sac de pommes de terre et leur nombre.
\item[c)] Le volume d'eau à l'intérieur d'un verre cylindrique 
et la hauteur d'eau à l'intérieur. 
\item[d)] Le volume d'eau à l'intérieur d'un verre à pied
et la hauteur d'eau à l'intérieur. 
\item[e)] La taille d'un individu et son âge.
\item[f)] La température moyenne d'une ville et son altitude. 
\item[g)] Le nombre d'habitants d'un pays et sa superficie. 
\end{itemize}



\section*{Exercice 2}
{Dire si les tableaux suivants sont de tableaux de proportionnalité. Justifier.}

\begin{enumerate}
	\item $\renewcommand{\arraystretch}{1}
\begin{array}{|c|c|c|}
\hline
\phantom{000}21\phantom{000} & \phantom{000}27\phantom{000} & \phantom{000}15\phantom{000}\\
\hline
7 & 9 & 5\\
\hline
\end{array}
\renewcommand{\arraystretch}{1}$

	\item $\renewcommand{\arraystretch}{1}
\begin{array}{|c|c|c|}
\hline
\phantom{000}9\phantom{000} & \phantom{000}5\phantom{000} & \phantom{000}8\phantom{000}\\
\hline
7 & 3 & 6\\
\hline
\end{array}
\renewcommand{\arraystretch}{1}$

	\item $\renewcommand{\arraystretch}{1}
\begin{array}{|c|c|c|}
\hline
\phantom{000}5{,}5\phantom{000} & \phantom{000}1{,}5\phantom{000} & \phantom{000}5\phantom{000}\\
\hline
16{,}5 & 4{,}5 & 15\\
\hline
\end{array}
\renewcommand{\arraystretch}{1}$

	\item $\renewcommand{\arraystretch}{1}
\begin{array}{|c|c|c|}
\hline
\phantom{000}9\phantom{000} & \phantom{000}7\phantom{000} & \phantom{000}5\phantom{000}\\
\hline
13 & 11 & 9\\
\hline
\end{array}
\renewcommand{\arraystretch}{1}$
	\item $\renewcommand{\arraystretch}{1}
\begin{array}{|c|c|c|}
\hline
\phantom{000}10\phantom{000} & \phantom{000}12\phantom{000} & \phantom{000}14\phantom{000}\\
\hline
5 & 6 & 7\\
\hline
\end{array}
\renewcommand{\arraystretch}{1}$

	\item $\renewcommand{\arraystretch}{1}
\begin{array}{|c|c|c|}
\hline
\phantom{000}7\phantom{000} & \phantom{000}8\phantom{000} & \phantom{000}6\phantom{000}\\
\hline
4 & 5 & 3\\
\hline
\end{array}
\renewcommand{\arraystretch}{1}$

	\item $\renewcommand{\arraystretch}{1}
\begin{array}{|c|c|c|}
\hline
\phantom{000}9\phantom{000} & \phantom{000}7\phantom{000} & \phantom{000}8\phantom{000}\\
\hline
13 & 11 & 12\\
\hline
\end{array}
\renewcommand{\arraystretch}{1}$

	\item $\renewcommand{\arraystretch}{1}
\begin{array}{|c|c|c|}
\hline
\phantom{000}9\phantom{000} & \phantom{000}8\phantom{000} & \phantom{000}6\phantom{000}\\
\hline
63 & 56 & 42\\
\hline
\end{array}
\renewcommand{\arraystretch}{1}$

	\item $\renewcommand{\arraystretch}{1}
\begin{array}{|c|c|c|}
\hline
\phantom{000}54\phantom{000} & \phantom{000}36\phantom{000} & \phantom{000}42\phantom{000}\\
\hline
9 & 6 & 7\\
\hline
\end{array}
\renewcommand{\arraystretch}{1}$

	\item $\renewcommand{\arraystretch}{1}
\begin{array}{|c|c|c|}
\hline
\phantom{000}6\phantom{000} & \phantom{000}5\phantom{000} & \phantom{000}9\phantom{000}\\
\hline
4 & 3 & 7\\
\hline
\end{array}
\renewcommand{\arraystretch}{1}$

	\item $\renewcommand{\arraystretch}{1}
\begin{array}{|c|c|c|}
\hline
\phantom{000}5\phantom{000} & \phantom{000}6\phantom{000} & \phantom{000}9\phantom{000}\\
\hline
11 & 12 & 15\\
\hline
\end{array}
\renewcommand{\arraystretch}{1}$

	\item $\renewcommand{\arraystretch}{1}
\begin{array}{|c|c|c|}
\hline
\phantom{000}7\phantom{000} & \phantom{000}5\phantom{000} & \phantom{000}8\phantom{000}\\
\hline
28 & 20 & 32\\
\hline
\end{array}
\renewcommand{\arraystretch}{1}$

	\item $\renewcommand{\arraystretch}{1}
\begin{array}{|c|c|c|}
\hline
\phantom{000}54\phantom{000} & \phantom{000}63\phantom{000} & \phantom{000}45\phantom{000}\\
\hline
6 & 7 & 5\\
\hline
\end{array}
\renewcommand{\arraystretch}{1}$
\end{enumerate}

\subsection*{Exercice 3}

1) On achète $450$ grammes de carottes à $3$ euros le kilogramme, combien paie-t-on ? 

2) Un paquet de cent stylos coûte $23$ euros, et un paquet de $50$ 
stylos coûte $12$ euros, les prix sont-ils proportionnels ? Lequel est le plus avantageux ?

\subsection*{Exercice 4}

Dans les situations suivantes, reconnaître celles qui sont proportionnelles. 

a) \begin{tabular}{|l | c |c | c| c|c| }\hline
Département & Ain & Gard & Loire & Rhône & Val-de-Marne\\
\hline
Nombre d'habitants & 650 000 & 750 000  & 770 000 & 1 880 000 &1 410 000 \\
Circonscriptions & 5 & 6 & 6 & 14 & 11 \\ \hline
\end{tabular} 

 \ \\ 
 
b) \begin{tabular}{|l | c | c | c | c | c}
\hline
Rayon d'un cercle & 1 & 2,5 &  5 & 10 \\
\hline
Périmètre & 3,14 &7,85 & 15,7& 31,4\\ \hline

\end{tabular}

 \ \\ 
 
 
c) \begin{tabular}{|l | c | c | c | c | c}
\hline
Rayon $r$ & 1 & 2,5 &  5 & 10 \\
\hline
Aire d'un disque de rayon $r$ & 3,14 & 19,625 & 78, 5& 314\\ \hline

\end{tabular}

 \ \\ 
 
d) \begin{tabular}{|l | c | c | c | c | c}
\hline
Longueur $r$ & 1 & 2 &  5 & 6 \\
\hline
Volume d'un cube de côté $r$ & 1 & 8 & 125 & 216 \\ \hline
Volume d'une sphère de rayon $r$ & 4,2 & 33, 6 & 525& 907,2\\ \hline

\end{tabular}
 
 
 \subsection*{Exercice 5}

Compléter les tableaux de proportionnalité suivants, et exprimer le coefficient de proportionnalité. 

a) \begin{tabular}{ | c |c | c| c|c| }\hline
 & 18  & 4 & 5 & \\
 \hline
 3 & 6 &  &  & 7,5 \\ \hline
\end{tabular} 

 \ \\ 
 
b) \begin{tabular}{|c | c | c | c | c | c}
\hline
2 & 5,5  & 7,5 &  5 &  \\
\hline
 & 1,1 & & &203 \\ \hline

\end{tabular}

 \ \\ 
 
 
c) \begin{tabular}{|c | c | c | c | c | c}
\hline
&  & 5 &   & 11      \\
\hline
6& 36 &  & 94  & 131  \\ \hline

\end{tabular}

 \ \\ 
 
d) \begin{tabular}{|c | c | c | c | c | c}
\hline
4 &  & 19 &   &  \\
\hline
& 99 &  & 62,7 & 46,2 \\ \hline

\end{tabular}



\subsection*{Exercice 6}
Identifier sur les graphiques suivants s'ils représentent ou non des situations de proportionnalité. \\
a)\begin{tikzpicture}
\draw[->](-1, 0) -- (4,0);
\draw[->](0,-1) -- (0,6);
\fill (2, 5) circle (0.1) (0.9, 2.25) circle (0.1) 
(0.5,1.25) circle (0.1) (-0.2, -0.5) circle (0.1);
\end{tikzpicture}\ \ \ \ \ \ \ \ \ b)
\begin{tikzpicture}
\draw[->](-1, 0) -- (6,0);
\draw[->](0,-1) -- (0,6);
\fill (2, 3) circle (0.1) (1, 2.5) circle (0.1) 
(0.5,2.25) circle (0.1) (4,4 ) circle (0.1);
\end{tikzpicture}\\
c)\begin{tikzpicture}
\draw[->](-1, 0) -- (6,0);
\draw[->](0,-1) -- (0,6);
\fill (2, 4) circle (0.1) (0.9, 1.8) circle (0.1) 
(0.5,1) circle (0.1) (1.4, 2.4) circle (0.1);
\end{tikzpicture}\ \ \ \ \ \ \ \ \ d)
\begin{tikzpicture}
\draw[->](-1, 0) -- (6,0);
\draw[->](0,-1) -- (0,6);
\fill (4, 1.33) circle (0.1) (3,1) circle (0.1) 
(1.5, .5) circle (0.1) (6, 2 ) circle (0.1);
\end{tikzpicture}






 	\end{document}
