\documentclass[12 pt]{extarticle}

	\usepackage[frenchb]{babel}
	\usepackage[utf8]{inputenc}  
	\usepackage[T1]{fontenc}
	\usepackage{amssymb}
	\usepackage[mathscr]{euscript}
	\usepackage{stmaryrd}
	\usepackage{amsmath}
	\usepackage{tikz}
	\usepackage[all,cmtip]{xy}
	\usepackage{amsthm}
	\usepackage{varioref}
	\usepackage{geometry}
	\geometry{a4paper}
	\usepackage{lmodern}
	\usepackage{hyperref}
	\usepackage{array}
	 \usepackage{fancyhdr}
	 \usepackage{float}
	\pagestyle{fancy}
	\theoremstyle{plain}
	\fancyfoot[C]{\thepage} 
	\fancyhead[L]{Fiche d'exercices}
	\fancyhead[R]{2022-2023}
	
	
\renewcommand{\theenumi}{\alph{enumi})}

\newlength{\taillecellule}
\setlength{\taillecellule}{2cm}
\newcolumntype{C}{@{}>{\centering\arraybackslash}p{\taillecellule}@{}}
\usetikzlibrary{calc}
\usepackage{pstricks,multido}
\usepackage{arrayjob}
\usepackage{calc,xlop}

\newcounter{AlphNode}
\renewcommand*{\theAlphNode}{\Alph{AlphNode}}

	\title{Exercices Chapitre 4}
	\date{}
	\begin{document}

\begin{center}{\Large Chapitre 4 - Triangles égaux}\\
 \end{center} 
 
 \subsection*{Exercice 1}
 \begin{figure}[H]
 \center
 \begin{tikzpicture}[scale=1.5]
    \foreach \x in {0,1,2,3,4}
    \foreach \y in {0,1,2,3,4}
    {
    \draw (\x,\y) -- ++(0, .1); 
    \draw (\x,\y) -- ++(0, -.1); 
    \draw (\x,\y) -- ++(.1,0); 
    \draw (\x,\y) -- ++(-.1,0); 
    \def\lettre{(\x+1)+5*(\y)};
    \pgfmathsetcounter{AlphNode}{Mod(\lettre,26)};
  \draw (\x, \y) ++ (.12,.2)node {\scriptsize\theAlphNode};
    }
  \end{tikzpicture}
 \end{figure}
  
  \begin{enumerate}
  \item Citer un triangle égal au triangle $AHL$. 
  \item Citer trois triangles égaux au triangle $AMP$. 
  \item Donner tous les triangles égaux au triangle $PVL$ ayant $A$ comme sommet.
   \item Donner tous les triangles égaux au triangle $PVL$ ayant $M$ comme sommet.
  \item Combien y a-t-il de triangles égaux à $PVI$ ayant $E$ comme sommet ? 
  \end{enumerate}
 
 	\end{document}
