\documentclass[14 pt]{extarticle}

	\usepackage[frenchb]{babel}
	\usepackage[utf8]{inputenc}  
	\usepackage[T1]{fontenc}
	\usepackage{amssymb}
	\usepackage[mathscr]{euscript}
	\usepackage{stmaryrd}
	\usepackage{amsmath}
	\usepackage{tikz}
	\usepackage[all,cmtip]{xy}
	\usepackage{amsthm}
	\usepackage{varioref}
	\usepackage{geometry}
	\geometry{a4paper}
	\usepackage{lmodern}
	\usepackage{float}
	\usepackage{hyperref}
	\usepackage{array}
	 \usepackage{fancyhdr}
\renewcommand{\theenumi}{\alph{enumi})}
	\pagestyle{fancy}
	\theoremstyle{plain}
	\fancyfoot[C]{} 
	\fancyhead[L]{Contrôle}
	\fancyhead[R]{24 novembre 2022}\geometry{
 a4paper,
 total={170mm,257mm},
 left=20mm,
 top=20mm,
 }
	
	
	\title{Contrôle Chapitre 3}
	\date{}
	\begin{document}

\begin{center}{\large Contrôle Chapitre 3}\\ 
 \end{center}
 
 
 \subsection*{Exercice 1 (5 points)}
 
 Donnez le signe des calculs suivants sans calculer le résultat : 
 \begin{enumerate}
 \item $ -5323,131 - 9,0132414139$, 
 \item $ - 324399239 + 13412324312414314$,
 \item $ (-1224)  \div (-2)$, 
 \item $ (-132843 ) \times (-124)$, 
 \item $(1343) \div (-14).$
 \end{enumerate}

\subsection*{Exercice 2 (7 points)}

Effectuez les calculs suivants : 
\begin{enumerate}
\item $ 2 + (-7) + 3$, 
\item $ 5 - (- 3)  - 5 $, 
\item $ 4 - (7 - (-2) ) + (-2) - (5 -12)$, 
\item $ (-3) \times 2 $, 
\item $ (3 - 6) \times (7- 10)$, 
\item $ (16-10) \div (-1 - 1)$, 
\item $( 2 - ( -8) ) \div ( -6 + 8)$. 
\end{enumerate}

\subsection*{Exercice 3 (6 points)}

Remplissez la pyramide de gauche comme une pyramide additive, et celle de droite comme une pyramide multiplicative. 
\begin{figure}[H]
\center
\begin{tikzpicture}[scale=.9,every node/.style={draw,minimum width=1.8cm,minimum height=.9cm}]
\draw (0,0)  node {};
\draw(-1,-1) node {} ++(2,0) node {8};
\draw(-2,-2) node {} ++(2,0) node {-2} ++(2,0) node {};
\draw(-3,-3) node {-3} ++(2,0) node {} ++(2,0) node {-1} ++(2,0) node {};
\end{tikzpicture}
\begin{tikzpicture}[scale=.9,every node/.style={draw,minimum width=1.8cm,minimum height=.9cm}]
\draw (0,0)  node {};
\draw(-1,-1) node {} ++(2,0) node {8};
\draw(-2,-2) node {} ++(2,0) node {-2} ++(2,0) node {};
\draw(-3,-3) node {-3} ++(2,0) node {} ++(2,0) node {-1} ++(2,0) node {};
\end{tikzpicture}\ 
\end{figure}



\subsection*{Exercice 4 (2 points)}


Effectuez les calculs suivants en détaillant les étapes : 
\begin{enumerate}
\item $16 \times (-7) + 4 \times (-5) - 12 \times 2$, 
\item $ 9 \div (-3) \times (-2)$
\end{enumerate}

%
%\subsection*{Exercice 5 (pour vous occuper à la fin)}
%
%Trouvez le nombre maximal de régions découpées par deux, puis trois, puis quatre cercles. Essayez de deviner le résultat pour cinq cercles.  Montrez qu'il est faux. 


 	\end{document}
