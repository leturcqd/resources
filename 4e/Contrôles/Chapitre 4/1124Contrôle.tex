\documentclass[14 pt]{extarticle}

	\usepackage[frenchb]{babel}
	\usepackage[utf8]{inputenc}  
	\usepackage[T1]{fontenc}
	\usepackage{amssymb}
	\usepackage[mathscr]{euscript}
	\usepackage{stmaryrd}
	\usepackage{amsmath}
	\usepackage{tikz}
	\usepackage[all,cmtip]{xy}
	\usepackage{amsthm}
	\usepackage{varioref}
	\usepackage{geometry}
	\geometry{a4paper}
	\usepackage{lmodern}
	\usepackage{float}
	\usepackage{hyperref}
	\usepackage{array}
	 \usepackage{fancyhdr}
\renewcommand{\theenumi}{\alph{enumi})}
	\pagestyle{fancy}
	\theoremstyle{plain}
	\fancyfoot[C]{} 
	\fancyhead[L]{Contrôle}
	\fancyhead[R]{15 décembre 2022}\geometry{
 a4paper,
 total={170mm,257mm},
 left=20mm,
 top=20mm,
 }
	
	
	\title{Contrôle Chapitre 4}
	\date{}
	\begin{document}

\begin{center}{\large Contrôle Chapitre 4}\\ 
 \end{center}
 
 
 \subsection*{Exercice 1 (5 points)}
 
Les triangles suivants sont-ils égaux ? \textbf{(Le dessin n'est pas à l'échelle)}.
\begin{enumerate}
\item
\begin{figure}[H]
\center
 \begin{tikzpicture}
\draw (0,0) -- (3,2) -- (5,1) -- (0,0); 
\end{tikzpicture}
 et  \ \ \ \ \ \ \ 
 \begin{tikzpicture}
\draw (0,0) -- (3,2) -- (5,1) -- (0,0); 
\end{tikzpicture}
\end{figure}
 \item
\begin{figure}[H]
\center
 \begin{tikzpicture}
\draw (0,0) -- (3,2) -- (5,1) -- (0,0); 
\end{tikzpicture}
 et  \ \ \ \ \ \ \ 
 \begin{tikzpicture}
\draw (0,0) -- (3,2) -- (5,1) -- (0,0); 
\end{tikzpicture}
\end{figure}
\item
\begin{figure}[H]
\center
 \begin{tikzpicture}
\draw (0,0) -- (3,2) -- (5,1) -- (0,0); 
\end{tikzpicture}
 et  \ \ \ \ \ \ \ 
 \begin{tikzpicture}
\draw (0,0) -- (3,2) -- (5,1) -- (0,0); 
\end{tikzpicture}
\end{figure}
\end{enumerate}

\subsection*{Exercice 2}
 
 Trouver la mesure manquante sachant que les deux triangles de la question sont égaux. Justifier la réponse.

\begin{enumerate}
\item
\begin{figure}[H]
\center
 \begin{tikzpicture}
\draw (0,0) -- (3,2) -- (5,1) -- (0,0); 
\end{tikzpicture}
 et  \ \ \ \ \ \ \ 
 \begin{tikzpicture}
\draw (0,0) -- (3,2) -- (5,1) -- (0,0); 
\end{tikzpicture}
\end{figure}
 \item
\begin{figure}[H]
\center
 \begin{tikzpicture}
\draw (0,0) -- (3,2) -- (5,1) -- (0,0); 
\end{tikzpicture}
 et  \ \ \ \ \ \ \ 
 \begin{tikzpicture}
\draw (0,0) -- (3,2) -- (5,1) -- (0,0); 
\end{tikzpicture}
\end{figure}
\end{enumerate}

\subsection*{Exercice 3}
 \begin{enumerate}
\item  Tracer un losange $ABCD$ avec $AC=4$ cm et $BD = 3$ cm. On note $O$ le point d'intersection des diagonales $[AC]$ et $[BD]$. \item Démontrer que les triangles $ABC$ et $BCD$ sont égaux. \end{enumerate}
 
\subsection*{Exercice 4}
Tracé : 
\begin{enumerate}

\item Tracer un carré $ABCD$, de côté $4$ cm. 
\item Placer un point $E\in[AB]$ tel que $AE=3$ cm, un point $F\in[AD]$ tel que $AF=1$ cm, et un point $G\in [CD]$ tels que $DG=1$ cm. 
\end{enumerate}
Raisonnement : 
\begin{enumerate}

\item Montrer que les triangles $AEF$ et $DFG$ sont égaux. 

\item En déduire que $EFG$ est un triangle isocèle rectangle. 
\end{enumerate}


\subsection*{Exercice 5}
Calculer : 
\begin{enumerate}
\item $2 + (-5) \times 4 + (-2)$, 
\item $24 \div (-2) \times (-3)$
\item $8 \div (-2) \div 2$. 
\end{enumerate}
 	\end{document}
