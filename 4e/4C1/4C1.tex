\documentclass[12 pt]{extarticle}

	\usepackage[frenchb]{babel}
	\usepackage[utf8]{inputenc}  
	\usepackage[T1]{fontenc}
	\usepackage{amssymb}
	\usepackage[mathscr]{euscript}
	\usepackage{stmaryrd}
	\usepackage{amsmath}
	\usepackage{tikz}
	\usepackage[all,cmtip]{xy}
	\usepackage{amsthm}
	\usepackage{varioref}
	\usepackage{geometry}
	\geometry{a4paper}
	\usepackage{lmodern}
	\usepackage{hyperref}
	\usepackage{array}
	 \usepackage{fancyhdr}

	\pagestyle{fancy}
	\theoremstyle{plain}
	\fancyfoot[C]{\thepage} 
	\fancyhead[L]{Fiche d'exercices}
	\fancyhead[R]{2022-2023}
	
	
	\title{Exercices Chapitre 1}
	\date{}
	\begin{document}

\begin{center}{\Large Chapitre 1 - Proportionnalité}\\
 \end{center} 

\subsection*{Exercice 1}

Parmi les situations suivantes, reconnaître lesquelles sont des situations de proportionnalité : \begin{itemize}
\item[a)] Le poids d'un sac de pommes de terre et leur poids.
\item[b)] Le prix d'un sac de pommes de terre et leur nombre.
\item[c)] Le volume d'eau à l'intérieur d'un verre cylindrique 
et la hauteur d'eau à l'intérieur. 
\item[d)] Le volume d'eau à l'intérieur d'un verre à pied
et la hauteur d'eau à l'intérieur. 
\item[e)] La taille d'un individu et son âge.
\item[f)] La température moyenne d'une ville et son altitude. 
\item[g)] Le nombre d'habitants d'un pays et sa superficie. 
\end{itemize}

\subsection*{Exercice 2}

1) On achète $450$ grammes de carottes à $3$ euros le kilogramme, combien paie-t-on ? 

2) Un paquet de cent stylos coûte $23$ euros, et un paquet de $50$ 
stylos coûte $12$ euros, les prix sont-ils proportionnels ? Lequel est le plus avantageux ?

\subsection*{Exercice 3}

Dans les situations suivantes, reconnaître celles qui sont proportionnelles. 

a) \begin{table}[!H]{ll}
\\
\end{table} 


 	\end{document}
