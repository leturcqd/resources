\documentclass[14 pt]{extarticle}

	\usepackage[frenchb]{babel}
	\usepackage[utf8]{inputenc}  
	\usepackage[T1]{fontenc}
	\usepackage{amssymb}
	\usepackage[mathscr]{euscript}
	\usepackage{stmaryrd}
	\usepackage{amsmath}
	\usepackage{tikz}
	\usepackage[all,cmtip]{xy}
	\usepackage{amsthm}
	\usepackage{varioref}
	\usepackage{geometry}
	\geometry{a4paper}
	\usepackage{lmodern}
	\usepackage{hyperref}
	\usepackage{array}
	 \usepackage{fancyhdr}
	 \usepackage{float}
\renewcommand{\theenumi}{\alph{enumi})}
	\pagestyle{fancy}
	\theoremstyle{plain}
	\fancyfoot[C]{} 
	\fancyhead[L]{Interrogation}
	\fancyhead[R]{16 janvier 2022}\geometry{
 a4paper,
 total={170mm,257mm},
 left=20mm,
 top=20mm,
 }
	
	
	\title{Interrogation chapitre 5}
	\date{}
	\begin{document}

\begin{center}{\Large Interrogation chapitre 5}\\ 
 \end{center}
 Nom : \ldots\ldots\ldots\\
 Prénom : \ldots\ldots\ldots
 
 \subsection*{Exercice 1 (4 points)}
 Complétez directement sur l'énoncé : 
 \begin{enumerate}
 \item $\displaystyle\frac15 + \frac65 = \frac{\color{white}7}{\color{white}5}$;
 \item $\displaystyle\frac34 - \frac74 = \frac{\color{white}-4}{\color{white}4} = \color{white} -1  $;
 \item $\displaystyle\frac38 - \frac78  + \frac58 = \frac{\color{white}1}{\color{white}8}$;
 \item $\displaystyle\frac13 - \left(\frac23 + \frac43\right) = \frac{\color{white}-5}{\color{white}3}$;
 \end{enumerate}
 
 \subsection*{Exercice 2 (6 points)}
 Remplir les deux pyramides additives suivantes (chaque case est la somme des deux cases sur lesquelles elle repose). 


\begin{figure}[H]
\center
\begin{tikzpicture}[scale=1.4,every node/.style={draw,minimum width=2.8cm,minimum height=1.4cm}]
\draw (0,0)  node {$\displaystyle\color{white}\frac{35}{18}$};
\draw(-1,-1) node {$\displaystyle\color{white} \frac56 $} ++(2,0) node {$\displaystyle\color{white}\frac{10}9$};
\draw(-2,-2) node {$\displaystyle \frac16$} ++(2,0) node {$\displaystyle\frac23$} ++(2,0) node {$\displaystyle\frac49$};
\end{tikzpicture}\ \begin{tikzpicture}[scale=1.4,every node/.style={draw,minimum width=2.8cm,minimum height=1.4cm}]
\draw (0,0)  node {$\displaystyle\color{white}\frac{87}{40}$};
\draw(-1,-1) node {$\displaystyle \frac{11}8$} ++(2,0) node {$\displaystyle \frac{8}{10}$};
\draw(-2,-2) node {$\displaystyle \frac78$} ++(2,0) node {$\color{white}{\displaystyle\frac12}$} ++(2,0) node {$\color{white}{\displaystyle\frac3{10}}$};
\end{tikzpicture}
\end{figure}


\subsection*{Exercice 3 (6 points)}

Effectuez sur votre feuille les calculs suivants, en détaillant vos étapes. 

\begin{enumerate}
\item $\displaystyle \frac14 - \frac15 = \color{white} \frac{5\times 1}{5\times4} - \frac{1\times4}{5\times 4} = \frac5{20}-\frac4{20} = \frac1{20}$ 

\item $\displaystyle \frac34 + \frac23 - \frac15
= \color{white} \frac{3\times15}{4\times15} + \frac{2\times20}{3\times 20}- \frac{1\times 12}{5\times12} = \frac{45}{60} + \frac{40}{60} - \frac{12}{60} = \frac{73}{60}$  

\item $\displaystyle \frac{14}{105} + \frac{12}{350} = \color{white} 
\frac{14\times 10}{105\times 10} + \frac{12\times 3}{350\times3} = \frac{140}{1050} + \frac{36}{1050} = \frac{176}{1050} = \frac{88}{525}$ 
\end{enumerate}

\subsection*{Bonus}
Calculez $\displaystyle \frac12 + \frac1{2 \times 2} + \frac1{2\times 2 \times 2} + \ldots + \frac1{2\times 2\times 2 \times 2 \times 2 \times 2 \times 2}= \color{white}\frac{127}{128}$.

\newpage 


\begin{center}{\Large Interrogation chapitre 4}\\ 
 \end{center}
 
 Nom : \ldots\ldots\ldots\\
 Prénom : \ldots\ldots\ldots
 
  \subsection*{Exercice 1 (4 points)}
 Complétez directement sur l'énoncé : 
 \begin{enumerate}
 \item $\displaystyle\frac37 + \frac57 =\frac{\color{white}8}{\color{white}7}$;
 \item $\displaystyle\frac98 - \frac{12}8 = \frac{\color{white}-3}{\color{white}8}$;
 \item $\displaystyle\frac34 - \frac74  + \frac54 = \frac{\color{white}1}{\color{white}4}$;
 \item $\displaystyle\frac15 - \left(\frac25 + \frac45\right) =\frac{\color{white}-5}{\color{white}5} = \color{white} -1$;
 \end{enumerate}
 
 \subsection*{Exercice 2 (6 points)}
 Remplir les deux pyramides additives suivantes (chaque case est la somme des deux cases sur lesquelles elle repose). 


\begin{figure}[H]
\center
\begin{tikzpicture}[scale=1.4,every node/.style={draw,minimum width=2.8cm,minimum height=1.4cm}]
\draw (0,0)  node {$\displaystyle\color{white}\frac{49}{24}$};
\draw(-1,-1) node {$\displaystyle\color{white}\frac78$} ++(2,0) node {$\displaystyle\color{white}\frac76$};
\draw(-2,-2) node {$\displaystyle \frac18$} ++(2,0) node {$\displaystyle\frac34$} ++(2,0) node {$\displaystyle\frac5{12}$};
\end{tikzpicture}\ \begin{tikzpicture}[scale=1.4,every node/.style={draw,minimum width=2.8cm,minimum height=1.4cm}]
\draw (0,0)  node {$\displaystyle\color{white}\frac{87}{40}$};
\draw(-1,-1) node {$\displaystyle \frac{11}8$} ++(2,0) node {$\displaystyle \frac{8}{10}$};
\draw(-2,-2) node {$\displaystyle \frac78$} ++(2,0) node {$\color{white}{\displaystyle\frac12}$} ++(2,0) node {$\color{white}{\displaystyle\frac3{10}}$};
\end{tikzpicture}
\end{figure}


\subsection*{Exercice 3 (6 points)}

Effectuez sur votre feuille les calculs suivants, en détaillant vos étapes. 

\begin{enumerate}
\item $\displaystyle \frac14 - \frac13 = \color{white} \frac{1\times3}{4\times3} - \frac{1\times4}{3\times 4} = \frac3{12}-\frac4{12} = \frac{-1}{12}$  

\item $\displaystyle \frac14 + \frac23 - \frac15 = \color{white} \frac{1\times15}{4\times15} + \frac{2\times20}{3\times 20}- \frac{1\times 12}{5\times12} = \frac{15}{60} + \frac{40}{60} - \frac{12}{60} = \frac{43}{60}$   

\item $\displaystyle \frac{12}{105} + \frac{14}{350} = = \color{white} 
\frac{12\times 10}{105\times 10} + \frac{14\times 3}{350\times3} = \frac{120}{1050} + \frac{42}{1050} = \frac{162}{1050} = \frac{81}{525} = \frac{27}{175}$ 
\end{enumerate}

\subsection*{Bonus}
Calculez $\displaystyle \frac12 + \frac1{2 \times 2} + \frac1{2\times 2 \times 2} + \ldots + \frac1{2\times 2\times 2 \times 2 \times 2 \times 2 \times 2}$.
 	\end{document}
