\documentclass[14 pt]{extarticle}

	\usepackage[frenchb]{babel}
	\usepackage[utf8]{inputenc}  
	\usepackage[T1]{fontenc}
	\usepackage{amssymb}
	\usepackage[mathscr]{euscript}
	\usepackage{stmaryrd}
	\usepackage{amsmath}
	\usepackage{tikz}
	\usepackage[all,cmtip]{xy}
	\usepackage{amsthm}
	\usepackage{varioref}
	\usepackage{geometry}
	\geometry{a4paper}
	\usepackage{lmodern}
	\usepackage{hyperref}
	\usepackage{array}
	 \usepackage{fancyhdr}
	 \usepackage{float}
\renewcommand{\theenumi}{\alph{enumi})}
	\pagestyle{fancy}
	\theoremstyle{plain}
	\fancyfoot[C]{} 
	\fancyhead[L]{Devoir maison}
	\fancyhead[R]{pour le 7 novembre 2022}\geometry{
 a4paper,
 total={170mm,257mm},
 left=20mm,
 top=20mm,
 }
	
	
	\title{DM}
	\date{}
	\begin{document}

\begin{center}{\Large Devoir Maison}\\ 
 \end{center}
 \subsection*{Exercice 1 }
 
Pour les calculs suivants, numérotez les opérations, puis effectuez-les étape par étape. \begin{enumerate}
\item $2+3\times 5 - 4$
\item $6\times 4 \div 3 \times 2$
\item $30 \div (3 + 4\times 5 \div 2+2)$
\item $2 + 3 - 5 + 4 - 2 \times (4 - 2 + 2) \div 2$
\item $(4 - (3\times 6 \div 9)) \times (2 + (3\times 4 \div 2+2))\div 2 + 3$
\end{enumerate}
 
 \subsection*{Exercice 2 }
 
 Remplissez le tableau suivant pour que toutes les lignes soient proportionnelles entre elles. 
\[
\begin{array}{|c|c|c|c|c|}
\hline
\phantom{000}     \phantom{1}   \phantom{000} 
& \phantom{000}   4   \phantom{000} & 
\phantom{000}     5   \phantom{000} & 
\phantom{000}    \phantom{ 8}   \phantom{000} & 
\phantom{000}     9  \phantom{000}
\\\hline
\phantom{000}     7    \phantom{000} 
& \phantom{000}   28   \phantom{000} & 
\phantom{000}    \phantom{ 35}   \phantom{000} & 
\phantom{000}     56   \phantom{000} & 
\phantom{000}     \phantom{63}   \phantom{000}
\\\hline
\phantom{000}     21   \phantom{000} 
& \phantom{000}  \phantom{ 84}   \phantom{000} & 
\phantom{000}    \phantom{ 105}     \phantom{000} & 
\phantom{000}     \phantom{168}   \phantom{000} & 
\phantom{000}     \phantom{189}   \phantom{000}
\\\hline
\phantom{000}     \phantom{29}   \phantom{000} 
& \phantom{000}  \phantom{ 116}   \phantom{000} & 
\phantom{000}     145   \phantom{000} & 
\phantom{000}    \phantom{ 232}   \phantom{000} & 
\phantom{000}     \phantom{261}   \phantom{000}
\\\hline
\phantom{000}    \phantom{ 57 }  \phantom{000} 
& \phantom{000}   \phantom{228 }  \phantom{000} & 
\phantom{000}    \phantom{ 285   }\phantom{000} & 
\phantom{000}     \phantom{456 }  \phantom{000} & 
\phantom{000}     513   \phantom{000}
\\
\hline
\end{array}\]
 
 
\subsection*{Exercice 3} 

Le son se déplace dans l'air à la vitesse de $330$ mètres par seconde. En première approche, on peut considérer que la lumière se déplace instantanément. 
\begin{enumerate}
\item Convertir cette vitesse en kilomètres par heure. 
\item On entend le tonnerre $6$ secondes après avoir vu l'éclair. À quelle distance est tombée la foudre ? 
\item À quelle distance doit-on être d'un éclair pour l'entendre après une minute ? 
\item En réalité, la lumière parcourt environ $300\ 000$ kilomètres par seconde. Reprendre les questions précédentes, 
et vérifier que les résultats ne sont pas profondément changés. 
\end{enumerate}


\subsection*{Exercice 4}

On dispose de $126$ pains au chocolat et $310$ croissants. On veut former des paniers identiques avec ces viennoiseries. 

\begin{enumerate}
\item Combien de paniers peut-on former ? 
\item Quelle sera alors la composition de chaque panier ? 
\item On ajoute de plus $105$ pains aux raisins. Reprendre les deux questions précédentes. 
\end{enumerate}




 	\end{document}
