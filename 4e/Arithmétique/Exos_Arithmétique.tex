	\documentclass[12 pt]{article}

	\usepackage[frenchb]{babel}
	\usepackage[utf8]{inputenc}  
	\usepackage[T1]{fontenc}
	\usepackage{amssymb}
	\usepackage[mathscr]{euscript}
	\usepackage{stmaryrd}
	\usepackage{amsmath}
	\usepackage{tikz}
	\usepackage[all,cmtip]{xy}
	\usepackage{amsthm}
	\usepackage{varioref}
	\usepackage{geometry}
	\geometry{a4paper}
	\usepackage{lmodern}
	\usepackage{hyperref}
	\usepackage{array}
	 \usepackage{fancyhdr}
	 \usepackage{float}

	\pagestyle{fancy}
	\theoremstyle{plain}
	\fancyfoot[C]{\thepage} 
	\fancyhead[L]{Arithmétique}
	\fancyhead[R]{2022-2023}
	\newcounter{n}
	\numberwithin{n}{section}
	\newtheorem{theo}{Théorème}
	\labelformat{theo}{théorème}
	\newtheorem*{prop}{Propriété}
	\labelformat{prop}{propriété}
	\newtheorem*{cor}{Corollaire}
	\newtheorem{lm}{Lemme}
	\labelformat{lm}{lemme~#1}
	\newtheorem{hyp}[n]{Hypothèse}
	
	{\theoremstyle{definition}
	\newtheorem*{df}{Définition}
    \newtheorem*{rmq}{Remarque}
	\newtheorem*{nt}{Notation}
	\newtheorem*{voc}{Vocabulaire}
	\newtheorem*{meth}{Méthode}
	\newtheorem*{ex}{Exemple}
	\newtheorem*{exs}{Exemples}
	\newtheorem*{formule}{Formule}
	\newtheorem*{formules}{Formules}
	}

	\renewcommand\epsilon{\varepsilon}
	\renewcommand\phi{\varphi}
	\newcommand\R{\mathbb{R}}
	\newcommand\s{\mathbb{S}}
	
	
	
	\title{Exercices Arithmétique}
	\date{}
	\begin{document}

\begin{center}{\Large Exercices d'arithmétique}\\ 
 \end{center} 
\section*{Division euclidienne}

\subsection*{Exercice 1}

Effectuer les divisions euclidiennes de : \begin{enumerate}
\item $45$ par $4$ ; 
\item $59$ par $7$ ; 
\item $59$ par $9$ ; 
\item $63$ par $9$.
\end{enumerate}

\subsection*{Exercice 2}

Effectuer les divisions euclidiennes de : \begin{enumerate}
\item $63$ par $3$ ; 
\item $107$ par $7$ ; 
\item $1792$ par $8$ ; 
\item $624$ par $6$.
\end{enumerate}

\subsection*{Exercice 3}

Effectuer les divisions euclidiennes de : \begin{enumerate}
\item $747$ par $12$ ; 
\item $3949$ par $21$ ; 
\item $10240$ par $16$ ; 
\item $6399$ par $9$.
\end{enumerate}

\subsection*{Exercice 4}
On veut répartir une classe de $29$ élèves en groupes de $4$. \begin{enumerate}
\item Combien de groupes complets peut-on former ? 
\item Combien d'élèves resterait-il alors ? 
\item Donner une répartition possible si l'on autorise à former un groupe de plus de quatre personnes. 
\item Donner une répartition possible en autorisant seulement des groupes de $3$ ou $4$ personnes. 
\end{enumerate}

\subsection*{Exercice 5}

On veut trouver tous les nombres $N$ tels que les divisions euclidiennes de $43$ par $N$ et de $55$ par $N$ aient le même reste.
\begin{enumerate}
\item On note $R$ ce reste, $Q$ le quotient de la division euclidienne de $43$ par $N$. Écrire une expression littérale reliant
$Q$, $N$ et $R$.
\item On note $Q'$ le quotient de le division euclidienne de $55$ par $N$. Écrire une expression littérale reliant
$Q'$, $N$ et $R$.
\item En utilisant les deux questions précédentes, montrer que $N$ divise à la fois $55-R$ et $43-R$. 
\item En déduire que $N$ divise $12$. 
\item Écrire la liste des diviseurs de $12$. 
\item Vérifier que les nombres obtenus à la question précédente vérifient tous la propriété voulue au début de l'exercice.
\end{enumerate}

\subsection*{Exercice 6}

Reprendre l'exercice précédent pour vérifier qu'il existe exactement quatre nombres $N$ pour lesquels les divisions euclidiennes de $10 753$ et $10 768$ par $N$ ont le même reste.

\subsection*{Exercice 7}

Calculer les restes des divisions euclidiennes de \begin{enumerate}
\item $1 000 000 \ldots 000$ par $3$ ; 
\item $9 999 999 999 \ldots 999$ par $4$ ;
\item $1110987654321$ par $5$. 
\item $98765432109876543210987654321$ par $1000$. 
\end{enumerate}
Indication : il n'y a évidemment pas à calculer toute la division. Pour les deux premiers exemples, on peut essayer de regarder ce qui se passe pour les premières valeurs ($10$, $100$, $100$,\ldots, et $9$, $99$, $999$, \ldots) pour deviner ce qui se passera ensuite. 



\section*{Diviseurs et multiples}

\subsection*{Exercice 1}

Déterminer si : \begin{enumerate}
\item $72$ est multiple de $8$ ; 
\item $84$ divise $4$ ; 
\item $57$ admet $3$ comme diviseur ; 
\item $91$ est divisible par $13$. 
\end{enumerate}

\subsection*{Exercice 2}

\begin{enumerate}
\item Trouver un nombre multiple de $7$, de $9$ et de $21$. 
\item Quel est le plus petit tel nombre ? 
\end{enumerate}


\subsection*{Exercice 3}
Peut-on répartir $720$ élèves en classes de $30$ ? de $29$ ? 

\subsection*{Exercice 4}

On veut répartir $420$ livres équitablement sur des étagères. 
\begin{enumerate}
\item Le peut-on avec $18$ étagères ? 
\item Et avec $21$ étagères ? 
\end{enumerate}

\subsection*{Exercice 5}

Un cafetier doit composer des corbeilles identiques avec $36$ croissants et $24$ pains au chocolat (sans laisser de reste). 
\begin{enumerate}
\item Que peut-on dire du nombre $N$ de corbeilles, par rapport à $36$ et $24$ ? (Si ce n'est pas immédiat, noter $c$ le nombre de croissants par corbeille, et $p$ celui de pains au chocolat, et écrire une expression littérale entre $c$ et $N$ et une autre entre $p$ et $N$.)
\item Faire une liste des nombres à vérifier la propriété trouver à la question précédente. 
\item Vérifier que tous ces nombres donnent une solution, et donner la composition des corbeilles dans chacun des cas. (Indication : il y a six solutions). 
\end{enumerate}

\section*{Nombres premiers}
\subsection*{Exercice 1}

Les nombres suivants sont-ils premiers ? 
\[ 9 ; 13 ; 15 ; 21 ; 33; 37.\]

\subsection*{Exercice 2}

Examiner si le nombre $2\times 3 \times 5 + 7 x11$ est premier. (Indication : le calculer.)


\subsection*{Exercice 3}

Sans le calculer, dire pourquoi le nombre $2\times 3 \times 7 + 3 \times 5 \times 11$ n'est pas premier. 

\subsection*{Exercice 4}

\begin{enumerate}\item Trouver le plus petit nombre premier supérieur à $100$.
\item Même question avec $200$. 
\item Même question avec $300$. 
\end{enumerate}
\subsection*{Exercice 5}

Trouver le plus grand nombre premier inférieur à $750$. 

\subsection*{Exercice 6}

Faire la liste des nombres premiers inférieurs à $200$.

\subsection*{Exercice 7}

Vérifier qu'il n'y a aucun nombre premier entre $1328$ et $1360$.

\subsection*{Exercice 8}

\begin{enumerate}
\item Que dire du chiffre des unités d'un nombre premier supérieur à 10 ?
\item
Justifier la réponse. \item Est-ce une condition suffisante ? \end{enumerate}

\section*{Décomposition en facteurs premiers}
\subsection*{Exercice 1}

Décomposer en facteurs premiers les nombres suivants : 

\[ 14 ; 18 ; 23 ; 24 ; 26 ; 30 ; 45 ; 77 ; 735 ; 1 001. \]

\subsection*{Exercice 2}

Même exercice avec les nombres suivants : \[
14\times 23 ; 1800 ; 26 \times 30 \times 77 ; 14 014.\]

\subsection*{Exercice 3}

\begin{enumerate}
\item Calculer le produit $1\times 2 \times 3 \times 4 \times 5 \times 6 \times 7$. 
\item En déduire la décomposition de $5 040$ en facteurs premiers. 
\item En déduire la décomposition de $40 320$ en facteurs premiers. (Indication : par quoi faut-il multiplier $5 040$ pour trouver $40 320$ ?)
\end{enumerate}

\subsection*{Exercice 4}
Décomposer en facteurs premiers, sans les calculer, les produits suivants : 
\begin{enumerate}
\item $1\times 3 \times 5 \times 7 \times 9 \times 11 \times 13$ ;
\item $2 \times 4 \times 6 \times 8 \times 10 \times 12 \times 14$ ; 
\item $ 1\times 2 \times 3 \times \cdots \times 13 \times 14 \times 15$ ; 
\item $2 \times 4 \times 6 \times 8 \times \cdots \times 26 \times 28 \times 30$. 
\end{enumerate}

\subsection*{Exercice 5}
\begin{enumerate}
\item Décomposer en produit de nombres premiers les nombres 
$1 326$, $1 001$ et $31 416$. 
\item Décomposer en facteur premier le produit $1 326\times 1 001
\times 31 416$. 
\item En déduire que $1 326\times 1 001
\times 31 416$ est le carré d'un nombre entier, dont on donnera la décomposition en facteurs premier. 
\item Calculer $\sqrt{1 326\times 1 001
\times 31 416}$. 
\end{enumerate}

\subsection*{Exercice 6}

\begin{enumerate}
\item Décomposer en facteurs premiers les nombres $88$, $693$, et 
$1 617$. 
\item En déduire que $88 \times 693 \times 1 617$ est le cube d'un entier dont on donnera la décomposition en facteurs premiers. 
\item Calculer cet entier.
\end{enumerate}


\section*{Liste de diviseurs}
\subsection*{Exercice 1}
Établir la liste des diviseurs des nombres suivants : 
\[ 64 ; 100 ; 360 ; 504 ; 1 001 ; 2 048.\]

\subsection*{Exercice 2}

Même exercice avec les nombres suivants : 
\[ 131 ; 197 ; 397.\]

\subsection*{Exercice 3}


On veut partager une longueur de $168$ mm en longueurs égales, faisant un nombre entiers de millimètres. Combien de morceaux peut-on faire ? 

\subsection*{Exercice 4}

Établir la liste des diviseurs des nombres suivants : 
\[ 2^2 \times 3 \times 5^3  ; 2 \times 3^3 \times 7 ; 3^4 \times 11.\]

\subsection*{Exercice 5}

On considère un nombre dont la décomposition en facteurs premiers s'écrit $7^a \times 11^b$, où $a$ et $b$ sont des entiers. 
\begin{enumerate}
\item Quelle est la forme de la décomposition en facteur premier d'un diviseur de ce nombre ? 
\item En déduire le nombre de tels diviseurs en fonction de $a$ et de $b$. 
\end{enumerate}

\subsection*{Exercice 6}

Utiliser l'exercice précédent pour obtenir le nombre de diviseurs de $2^2\times 3^3\times 5$. 

\subsection*{Exercice 7}

Généralisant les exercices 5 et 6, déterminer le nombre de diviseurs de : \begin{enumerate}
\item $2^a$ (en fonction de $a$) ;
\item $2^a \times 3^b$ (en fonction de $a$ et de $b$) ;
\item $10^a$ (en fonction de $a$) ; 
\item $2^a\times 3^a\times 5^b$ (en fonction de $a$ et de $b$) ;
\item d'un nombre dont la décomposition en facteurs 
premiers est $p_1^{a_1}\times p_2^{a_2}\times\cdots\times p_n^{a_n}$. 
\end{enumerate}

\subsection*{Exercice 8 - pavages réguliers}

On veut paver le plan en utilisant uniquement des polygones réguliers à $N$ côtés de même taille. 

\begin{enumerate}
\item Rappeler la valeur de la somme des angles d'un triangle. 
\item En coupant un quadrilatère suivant une diagonale, en déduire la formule de la somme des angles d'un quadrilatère. 
\item En reliant tous les sommets d'un polygone régulier à $N$ côtés à son centre, combien de triangles forme-t-on ? 
\item En déduire soigneusement que la somme des angles d'un tel polygone est $(N-2)\times 180^o$. 
\item En déduire que chaque angle d'un polygone régulier à $N$ côtés mesure $\frac{N-2}N\times 180^o$.
\item On suppose qu'on a pu paver notre plan en utilisant des polygones à $N$ côtés. On suppose qu'autour de chaque sommet, il y a $M$ polygones. Montrer que $M\times \frac{N-2}N\times 180^o = 360^o$.
\item En déduire que $\frac{2\times N}{N-2}$ est un entier. 
\item Démontrer que $\frac{2\times N}{N-2} = 2 + \frac{4}{N-2}$. 
\item En déduire que $N-2$ doit diviser $4$. 
\item Quelles sont les valeurs possibles de $N$ ? (Indication : il y en a $3$.)
\item Tracer les trois pavages possibles. 
\end{enumerate}

\section*{Plus grand diviseur commun}

\subsection*{Exercice 1}

Calculer le plus grand diviseur commun de $720$ et $216$. 


\subsection*{Exercice 2}

Calculer le plus grand diviseur commun de $1 092$ et $228$. 


\subsection*{Exercice 3}

Dans un collège, toutes les classes ont la même taille. D'autre part, il y a $174$ élèves en $6^e$ et $145$ élèves en $5^e$. En déduire le nombre d'élèves par classe. 

\subsection*{Exercice 4}

Simplifier au maximum les fractions suivantes  : 
\[ \frac8{24} ; \frac{720}{216} ; \frac{228}{1092} ; \frac{100}{256}
; \frac{1 326}{31 416}.\]

\section*{Plus petit multiple commun}

\subsection*{Exercice 1}


\begin{enumerate}
\item Calculer le plus petit multiple commun de $720$ et $216$. 

\item Même question avec $228$ et $1 092$. 

\item En utilisant les exercices $1$ et $2$ de la partie précédente, 
comparer dans les deux exemples le produit du plus grand diviseur commun 
et du plus petit multiple commun avec le produit des deux nombres. 
Constater. 
\end{enumerate}

\subsection*{Exercice 2}

Donner la liste des multiples communs à $10$ et $12$ jusque $400$. 

\subsection*{Exercice 3}

\begin{enumerate}
\item Calculer le plus petit nombre à être multiple de tous les entiers de $1$ à $10$ inclus. 
\item Donner tous les nombres à au plus quatre chiffres à être multiples de tous les entiers de $1$ à $10$. 
\end{enumerate}

\subsection*{Exercice 4}

Réduire au même dénominateur les fractions suivantes, puis les ranger dans l'ordre croissant et les additionner. Réduire ensuite le résultat en une fraction irréductible.
\begin{enumerate}
\item $\frac16$ et $\frac1{15}$ ;
\item $\frac12$, $\frac13$ et $\frac16$ ;
\item $\frac14$, $\frac1{11}$ et $\frac5{44}$. 
\item $\frac57$, $\frac29$, et $\frac4{62}$. 
\end{enumerate}

	\end{document}
	


