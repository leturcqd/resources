\documentclass[14 pt]{extarticle}

	\usepackage[frenchb]{babel}
	\usepackage[utf8]{inputenc}  
	\usepackage[T1]{fontenc}
	\usepackage{amssymb}
	\usepackage[mathscr]{euscript}
	\usepackage{stmaryrd}
	\usepackage{amsmath}
	\usepackage{tikz}
	\usepackage[all,cmtip]{xy}
	\usepackage{amsthm}
	\usepackage{varioref}
	\usepackage{geometry}
	\geometry{a4paper}
	\usepackage{lmodern}
	\usepackage{hyperref}
	\usepackage{array}
	 \usepackage{fancyhdr}
\renewcommand{\theenumi}{\alph{enumi})}
	\pagestyle{fancy}
	\theoremstyle{plain}
	\fancyfoot[C]{} 
	\fancyhead[L]{Contrôle}
	\fancyhead[R]{1 février 2023}\geometry{
 a4paper,
 total={170mm,257mm},
 left=20mm,
 top=20mm,
 }
	
	
	\title{Interrogation chapitre 5}
	\date{}
	\begin{document}

\begin{center}{\Large Contrôle chapitre 5}\\ 
 \end{center}
 
 
  \subsection*{Exercice 1 (5 points)}
  
Sur votre copie, posez et effectuez les calculs suivants : \begin{enumerate}
\item $56,12 + 34, 914$
\item $56,12 - 34, 914$
\item $2 h 45 min + 2h 43 min$
\item $3 h 45 min - 2h 58 min$
\end{enumerate}  

\subsection*{Exercice 2 (5 points)}

En 2022, le prix de la tonne de lait a monté de $70$ centimes en janvier, de $1$ euro et $40$ centimes en février, de $2$ euros $30$ centimes en mars, et de $1$ euro $80$ centimes en avril, pour atteindre le prix de $48$ euros la tonne. Calculez le prix de la tonne de lait début janvier $2022$. 


\subsection*{Exercice 3 (5 points)}

Marc prend le train Paris-Marseille de $18h38$, qui met $3$ heures et $8$ minutes à atteindre Aix-en-Provence, puis arrive à $21h58$ à Marseille.
\begin{enumerate}
\item Combien de temps le train met-il à aller de Paris à Aix-en-Provence ? 
\item Combien de temps met-il à aller d'Aix-en-Provence à Marseille ? 
\end{enumerate} 




\subsection*{Exercice 4 (5 points)}

Jean va au marché pour acheter un morceau de viande à $9$ euros $21$ centimes, et des légumes pour un prix de $5$ euros $90$ centimes. En rentrant, il prend une baguette de pain. Arrivé chez lui, il lui reste $3$ euros et $68$ centimes. Sachant qu'il était parti avec un billet de $20$ euros, combien coûte la baguette ?
  
  
  \newpage 

\begin{center}{\Large Contrôle chapitre 5}\\ 
 \end{center}
  
    \subsection*{Exercice 1 (5 points)}
  
Sur votre copie, posez et effectuez les calculs suivants : \begin{enumerate}
\item $65,12 + 34, 914$
\item $65,12 - 34, 914$
\item $1 h 45 min + 2h 43 min$
\item $3 h 45 min - 2h 53 min$
\end{enumerate}  

\subsection*{Exercice 2 (5 points)}

En 2023, le prix de la tonne de lait a baissé de $2$ euros et $40$ centimes en janvier, de $3$ euros en février, de $2$ euros $70$ centimes en mars, et de $1$ euro en avril, pour atteindre le prix de $46$ euros $60$ la tonne. Calculez le prix de la tonne de lait début janvier $2023$. 


\subsection*{Exercice 3 (5 points)}

Marc prend le train Paris-Marseille de $14h38$, qui met $3$ heures et $8$ minutes à atteindre Aix-en-Provence, puis arrive à $17h58$ à Marseille.
\begin{enumerate}
\item Combien de temps le train met-il à aller de Paris à Aix-en-Provence ? 
\item Combien de temps met-il à aller d'Aix-en-Provence à Marseille ? 
\end{enumerate} 




\subsection*{Exercice 4 (5 points)}

Jean va au marché pour acheter un morceau de viande à $9$ euros $12$ centimes, et des légumes pour un prix de $5$ euros $90$ centimes. En rentrant, il prend une baguette de pain. Arrivé chez lui, il lui reste $3$ euros et $68$ centimes. Sachant qu'il était parti avec un billet de $20$ euros, combien coûte la baguette ?
  
 	\end{document}
