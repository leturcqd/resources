\documentclass[12 pt]{extarticle}

	\usepackage[frenchb]{babel}
	\usepackage[utf8]{inputenc}  
	\usepackage[T1]{fontenc}
	\usepackage{amssymb}
	\usepackage[mathscr]{euscript}
	\usepackage{stmaryrd}
	\usepackage{amsmath}
	\usepackage{tikz}
	\usepackage[all,cmtip]{xy}
	\usepackage{amsthm}
	\usepackage{varioref}
	\usepackage{geometry}
	\geometry{a4paper}
	\usepackage{lmodern}
	\usepackage{hyperref}
	\usepackage{array}
	 \usepackage{fancyhdr}
\renewcommand{\theenumi}{\alph{enumi})}
	\pagestyle{fancy}
	\theoremstyle{plain}
	\fancyfoot[C]{} 
	\fancyhead[L]{Contrôle}
	\fancyhead[R]{17 janvier 2023}\geometry{
 a4paper,
 total={170mm,257mm},
 left=20mm,
 top=20mm,
 }
	
	
	\title{Contrôle Chapitre 4}
	\date{}
	\begin{document}

\begin{center}{\Large Contrôle}\\ 
 \end{center}
 
 \subsection*{Exercice 1}
 \begin{enumerate}
 \item Construisez un segment $[AB]$ de longueur $3$ cm. 
 
 
 \item Construisez la perpendiculaire $(d)$ à $(AB)$ passant par $B$.  
 \item Placez un point $C$ sur $(d)$ tel que $BC=4$ cm. 
	
 \item Tracez le cercle $(\mathcal C)$ de centre $A$ passant par $C$.
 
 \item Quelle est la nature du triangle $ABC$ ? Démontrez votre réponse. 
 
 \item Tracez la perpendiculaire à $(AB)$ passant par $A$. 
 
 \item Démontrez que cette droite est parallèle à $(CB)$. 
 
 
 \end{enumerate}
 \subsection*{Exercice 2}
 
 \begin{enumerate}
 \item Tracez un segment $[AB]$ de longueur $5$ cm. 
 
 \item Tracez le cercle de centre $A$ passant par $B$. 
 
 \item Tracez le cercle de centre $B$ passant par $A$. 
 
 \item Placez un point $C$ dans l'intersection de ces deux cercles. 
 
 \item Quelle est la nature du triangle $ABC$ ? Démontrez votre réponse. 
 \end{enumerate}
 
 
 \subsection*{Exercice 3}
 
 \begin{enumerate}
 \item Tracez un triangle $ABC$ dont les côtés mesurent $4$ cm, $7$ cm et $5$ cm. 
 \item Tracez la médiatrice du côté $[AB]$. 
 \item Tracez la médiatrice du côté $[BC]$. 
 \item Tracez la médiatrice du côté $[AC]$. 
 \item Placez le point $O$ à l'intersection des trois médiatrices. 
 \item Tracez le cercle de centre $O$ passant par $A$. Que remarquez-vous ? \end{enumerate}
 

\newpage

\begin{center}{\Large Contrôle}\\ 
 \end{center}
 
 \subsection*{Exercice 1}
 \begin{enumerate}
 \item Construisez un segment $[AB]$ de longueur $4$ cm. 
 
 
 \item Construisez la perpendiculaire $(d)$ à $(AB)$ passant par $B$.  
 \item Placez un point $C$ sur $(d)$ tel que $BC=5$ cm. 
	
 \item Tracez le cercle $(\mathcal C)$ de centre $A$ passant par $C$.
 
 \item Quelle est la nature du triangle $ABC$ ? Démontrez votre réponse. 
 
 \item Tracez la parallèle à $(BC)$ passant par $A$. 
 
 \item Démontrez que cette droite est perpendiculaire à $(AB)$. 
 
 
 \end{enumerate}
 \subsection*{Exercice 2}
 
 \begin{enumerate}
 \item Tracez un segment $[AB]$ de longueur $4$ cm. 
 
 \item Tracez le cercle de centre $A$ passant par $B$. 
 
 \item Tracez le cercle de centre $B$ passant par $A$. 
 
 \item Placez un point $C$ dans l'intersection de ces deux cercles. 
 
 \item Quelle est la nature du triangle $ABC$ ? Démontrez votre réponse. 
 \end{enumerate}
 
 
 \subsection*{Exercice 3}
 
 \begin{enumerate}
 \item Tracez un triangle $ABC$ dont les côtés mesurent $4$ cm, $7$ cm et $5$ cm. 
 \item Tracez la médiatrice du côté $[AB]$. 
 \item Tracez la médiatrice du côté $[BC]$. 
 \item Tracez la médiatrice du côté $[AC]$. 
 \item Placez le point $O$ à l'intersection des trois médiatrices. 
 \item Tracez le cercle de centre $O$ passant par $A$. Que remarquez-vous ? \end{enumerate}
 




 
 	\end{document}
