	\documentclass[14pt]{extreport}
\usepackage{extsizes}
	\usepackage[frenchb]{babel}
	\usepackage[utf8]{inputenc}  
	\usepackage[T1]{fontenc}
	\usepackage{amssymb}
	\usepackage[mathscr]{euscript}
	\usepackage{stmaryrd}
	\usepackage{amsmath}
	\usepackage{tikz}
\usepackage{eurosym}
	\usepackage[all,cmtip]{xy}
	\usepackage{amsthm}
	\usepackage{varioref}
	\usetikzlibrary{patterns}
	\usepackage{float}
	\usepackage[ margin=1in]{geometry}
	\geometry{a4paper}
	\usepackage{lmodern}
	\usepackage{hyperref}
	\usepackage{array}
	\usepackage{easytable}
	 \usepackage{fancyhdr}\usepackage{longtable}

	\pagestyle{fancy}
	\theoremstyle{plain}
	\fancyfoot[C]{\empty} 
	\fancyhead[L]{Contrôle}
	\fancyhead[R]{3 juin 2024}
	
	
	\title{Contrôle chapitre 9}
	\date{}
	\begin{document}

\begin{center}{\Large Contrôle chapitre 9}\\ \textbf{Soignez votre présentation et votre rédaction.}\end{center}
 
 
 \subsection*{Exercice 1}
 
 Effectuer les divisions euclidiennes suivantes en donnant le quotient et le reste: 
 \begin{enumerate}
 \item $23$ par $4$ 
 \item $105$ par $5$ 
 \item $532 324$ par $10$
 \end{enumerate}
 
 \subsection*{Exercice 2}
 
 Jean veut partager $57$ pains au chocolat en $7$ personnes. Combien lui en restera-t-il à la fin ? 
 
 \subsection*{Exercice 3}
 	Effectuer les divisions décimales suivantes. 
 	\begin{enumerate}
 	\item $543\div 8$
 	\item $12 \div 25$
 	\item $102\div11$ (on s'arrêtera dès qu'un motif se répète.).
\end{enumerate} 	 
\subsection*{Exercice 4}

On doit répartir une somme de $1 250$ euros en huit personnes. Combien recevra chacun ? 

\subsection*{Exercice 5}

Calculez l'aire d'un huitième de disque de rayon $10$ cm. Donnez une valeur exacte, puis donnez une valeur approchée au millimètre carré. On utilisera $3,1416$ comme valeur approchée de $\pi$. 

\end{document}