\documentclass[12 pt]{extarticle}

	\usepackage[frenchb]{babel}
	\usepackage[utf8]{inputenc}  
	\usepackage[T1]{fontenc}
	\usepackage{amssymb}
	\usepackage[mathscr]{euscript}
	\usepackage{stmaryrd}
	\usepackage{amsmath}
	\usepackage{tikz}
	\usepackage[all,cmtip]{xy}
	\usepackage{amsthm}
	\usepackage{varioref}
	\usepackage{geometry}
	\geometry{a4paper}
	\usepackage{lmodern}
	\usepackage{hyperref}
	\usepackage{array}
	 \usepackage{fancyhdr}
\renewcommand{\theenumi}{\alph{enumi})}
	\pagestyle{fancy}
	\theoremstyle{plain}
	\fancyfoot[C]{} 
	\fancyhead[L]{Interrogation}
	\fancyhead[R]{2023-2024}\geometry{
 a4paper,
 total={170mm,257mm},
 left=20mm,
 top=20mm,
 }
	
	
	\title{Interrogation Chapitre 1}
	\date{}
	\begin{document}

\begin{center}{\Large Chapitre 1 - Rappels de  numération}\\ 
 \end{center}
  
  \subsection*{Exercice 1 (7 points)}
  
  On considère le nombre $908\ 323, 124$
  
  \begin{enumerate}
  \item Donner son chiffre des dizaines. 
  \item Donner son chiffre des millièmes. 
  \item Donner son chiffre des milliers. 
  \item Recopier la phrase suivante : « Le chiffre des \ldots est égal au chiffre des \ldots »
  \item Donner sa partie entière. 
  \item Donner sa partie décimale. 
  \item Écrire le nombre comme une fraction décimale. 
  \end{enumerate}
  
  \subsection*{Exercice 2 (5 points)}
  
  \begin{enumerate}
  \item Donner l'écriture décimale des nombres suivants. 
  \[ \frac{123}{10}, \frac{135}{1\ 000}, 
  12+ \frac{4}{100}\] 
  \item Écrire $\displaystyle\frac{1234}{10}$ comme la somme d'un entier et d'une fraction décimale inférieure à $1$. 
    \item Écrire $54,013$ comme la somme d'un entier et d'une fraction décimale inférieure à $1$. 
  
  \end{enumerate}
  
  \subsection*{Exercice 3 (4 points)}
  
  Calculer : \begin{enumerate}
  \item $0,12 \times 1\ 000$
  \item $213\div 100$
  \item $124,12\times 1\ 000$ 
  \item $765,23 \div 10\ 000$
  \end{enumerate}
  
  \subsection*{Exercice 4 (4 points)}

\begin{enumerate}
\item Donner un nombre dont le chiffre des dizaines est égal à la somme du chiffre des dixièmes et de celui des centaines.

\item Donner un nombre dont le chiffre des centaines est égal au produit du chiffre des millièmes et de celui des dizaines.

\item Donner un nombre à $6$ chiffres dont les chiffres sont distincts\footnote{non tous égaux} dont le chiffre des cent mille est égal à la somme des autres chiffres.
\item Donner le plus grand nombre possible pour la question précédente. 

\end{enumerate}

\newpage


\begin{center}{\Large Chapitre 1 - Rappels de  numération}\\ 
 \end{center}
  
  \subsection*{Exercice 1 (7 points)}
  
  On considère le nombre $756\ 987, 324$
  
  \begin{enumerate}
  \item Donner son chiffre des dizaines. 
  \item Donner son chiffre des millièmes. 
  \item Donner son chiffre des milliers. 
  \item Recopier la phrase suivante : « Le chiffre des \ldots est égal au chiffre des \ldots »
  \item Donner sa partie entière. 
  \item Donner sa partie décimale. 
  \item Écrire le nombre comme une fraction décimale. 
  \end{enumerate}
  
  \subsection*{Exercice 2 (5 points)}
  
  \begin{enumerate}
  \item Donner l'écriture décimale des nombres suivants. 
  \[ \frac{1234}{100}, \frac{159}{1\ 000}, 
  12+ \frac{4}{100}\] 
  \item Écrire $\displaystyle\frac{1239}{10}$ comme la somme d'un entier et d'une fraction décimale inférieure à $1$. 
    \item Écrire $54,897$ comme la somme d'un entier et d'une fraction décimale inférieure à $1$. 
  
  \end{enumerate}
  
  \subsection*{Exercice 3 (4 points)}
  
  Calculer : \begin{enumerate}
  \item $0,21 \times 1\ 000$
  \item $243\div 100$
  \item $124,21\times 1\ 000$ 
  \item $758,23 \div 10\ 000$
  \end{enumerate}
  
  \subsection*{Exercice 4 (4 points)}

\begin{enumerate}
\item Donner un nombre dont le chiffre des centaines est égal à la somme du chiffre des dixièmes et de celui des dizaines.

\item Donner un nombre dont le chiffre des centaines est égal au produit du chiffre des dixièmes et de celui des dizaines.

\item Donner un nombre à $6$ chiffres dont les chiffres sont distincts\footnote{non tous égaux} dont le chiffre des cent mille est égal à la somme des autres chiffres.
\item Donner le plus grand nombre possible pour la question précédente. 
\end{enumerate}


 	\end{document}
