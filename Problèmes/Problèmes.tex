\documentclass[12 pt]{extarticle}

	\usepackage[frenchb]{babel}
	\usepackage[utf8]{inputenc}  
	\usepackage[T1]{fontenc}
	\usepackage{amssymb}
	\usepackage[mathscr]{euscript}
	\usepackage{stmaryrd}
	\usepackage{amsmath}
	\usepackage{tikz}
	\usepackage[all,cmtip]{xy}
	\usepackage{amsthm}
	\usepackage{varioref}
	\usepackage{geometry}
	\geometry{a4paper}
	\usepackage{lmodern}
	\usepackage{hyperref}
	\usepackage{array}
	 \usepackage{fancyhdr}
	 \usepackage{float}
	\pagestyle{fancy}
	\theoremstyle{plain}
	\fancyfoot[C]{\thepage} 
	\fancyhead[L]{Fiche d'exercices}
	\fancyhead[R]{2022-2023}
	
	
	\title{Exercices Chapitre 7}
	\date{}
	\begin{document}

\begin{center}{\Large Exercices d'algèbre\footnote{Tirés du manuel de 4e Lebossé, Hémery, possibles fautes de recopie.}}\\
 \end{center} 
 
 \begin{enumerate}
 \item Trouver trois nombres entiers consécutifs\footnote{consécutifs : qui se suivent} dont la somme est égale à $57$. 
 \item La somme de cinq nombres impairs consécutifs est 85. Quels sont ces nombres ? 
 \item Un père a 29 ans ; son fils a 5 ans ; dans combien d'années l'âge du père sera-t-il le triple de l'âge du fils ? 
 \item Un père a 41 ans ; ses trois enfants sont âgés de 7, 9 et 13 ans. Dans combien d'années l'âge du père sera-t-il égal à la somme des âges de ses enfants ? 
 \item Trouver un nombre dont la somme des quotients par 5, 7 et 9 soit égale à 429. 
 \item En retranchant\footnote{retrancher : soustraire} 5 aux $\frac23$ d'un nombre, on trouve le même résultat qu'en ajoutant 2 aux $\frac35$ de ce nombre. Quel est ce nombre ? 
 \item Trouver un nombre dont le carré augmente de $189$ quand on augmente ce nombre de $7$. 
 \item Quel nombre faut-il ajouter aux deux termes de la fraction $\frac7{13}$ pour qu'elle devienne égale à $\frac23$ ? 
 \item Quel nombre faut-il retrancher aux deux termes de la fraction $\frac{17}{23}$ pour obtenir une fraction égale à $\frac58$ ? 
 \item Quel nombre faut-il ajouter aux deux termes de la fraction $\frac58$ et retrancher aux deux termes de la fraction $\frac34$ pour obtenir deux fractions égales ? 
 \item Une personne dispose de deux heures pour effectuer une promenade. Elle part en tramway à la vitesse moyenne de $12$ km à l'heure et revient à pied, à la vitesse moyenne de $4$ km à l'heure. À quelle distance du point de départ devra-t-elle quitter le tramway ? 
 \item Un épicier achète un certain nombre de bouteilles de vin pour $225$ francs. En vendant la bouteille $1,65$ francs, il gagnerait autant que ce qu'il perdrait s'il la vendait $1,35$ francs. Quel est le nombre des bouteilles de vins ? 
 \item Un capital est placé à $6$ \% pendant dix-huit mois. S'il était placé à 5 \% pendant deux ans, les intérêts augmenteraient de $450$ francs. Quel est ce capital ? 
 \item Deux capitaux dont l'un est les $\frac34$ de l'autre sont placés pendant dix-huit mois au même taux de 5 \%. La somme totale ainsi obtenue, capitaux et intérêts réunis, est 90 300 francs. Quels sont ces deux capitaux ? 
 \item Pour se rendre à son travail, un employé parcourt les $\frac34$ de la distance totale en autobus à la vitesse moyenne de $20$ km à l'heure, et le reste à pied à la vitesse moyenne de $5$ km à l'heure. Sachant qu'il met 21 minutes pour se rendre à son travail, quelle distance parcourt-il ? 
 \item À quelle heure entre $2$ heures et $3$ heures les aiguilles d'une montre sont-elles exactement l'une sur l'autre ? À quelle heure forment-elles un angle droit ? 
 \item Une automobile parcourt 207 km en 2 h 45 min. Elle perd les $\frac35$ de sa vitesse dans la traversée des agglomérations dont la longueur totale est $27$ km. Quelle est la vitesse moyenne, sur route, de cette automobile ? 
 \item Deux automobiles partent à la même heure d'une ville A pour une ville B, la première avec une vitesse moyenne de $80$ km à l'heure, la seconde avec une vitesse moyenne de 60 km à l'heure. Sachant que les heures d'arrivée sont 15:45 et 16:15, trouver la distance entre les villes A et B. 
 \item Une somme est partagée entre plusieurs personnes de la façon suivante : la première a 100 francs, plus le quart du reste ; la seconde a 200 francs plus le quatre du nouveau reste, et ainsi de suite. Il se trouve que toute les parts sont égales. Quelle est la somme partagée ? Quelle est la valeur d'une part et le nombre de parts ? 
 \item Une personne a placé les $\frac34$ d'un capital à 5 \% et le reste à 4,5\%. En 18 mois, la première partie rapporte 360 francs de plus que la seconde en un an. Quel est le capital total ? 
 %\item Un fil d'or et de cuivre pesant 83 grammes subit quand on le plonge dans l'eau une perte de poids de 7 grammes. Quelles sont les quantités d'or et de cuivre contenus dans ce fil sachant que les densités de l'or et du cuivre sont 19,5 et 8,8 ?
 \item Un mètre de drap coûte 7,2 francs de plus qu'un mètre de toile. Sachant que 10 m de drap et 12 m de toile coûtent ensemble 256,80 francs, trouver le prix du mètre de chaque étoffe. 
 \item Une fermière vend trois canards et quatre poulets pour 75 francs. Sachant qu'un canard et un poulet valent ensemble 21 francs, trouvez le prix d'un canard et celui d'un poulet. 
 \item Deux ouvriers gagnent ensemble 57 francs par jour. En un mois, le premier a travaillé 24 jours et le second 20 jours. Ils ont reçu ensemble 1260 francs. Quel est le salaire journalier de chacun d'eux ? 
 \item Un commerçant a vendu 5 m de toile et 10 m de drap pour 195 francs. Une seconde fois, il a vendu 27 m de toile et 23 m de drap pour obtenir 619 francs. Quel est le prix du mètre de chaque étoffe ? 
 \item Un bateau fait sur un fleuve le service entre deux localités A et B. Cette dernière est située à 25,200 km en aval de A. La vitesse du courant est de 3 km à l'heure et elle s'ajoute ou se retranche à la vitesse propre du bateau suivant qu'il descend ou remonte le courant. La durée du trajet de A à B est les trois quarts de celle du retour. Trouver la vitesse propre du bateau, puis calculer le retard dû au courant sur un aller-retour. 
 \item Un marchand a vendu une pièce de drap pour 1 224 francs, une pièce de soie pour 714 francs, et une pièce de toile pour 510 francs.La longueur de la pièce de drap surpasse de 3 mètres celle de la pièce de soie et est inférieure de 6 mètres à celle de la pièce de toile. Sachant qu'un mètre de drap coûte autant qu'un mètre de toile et un mètre de soie réunis, trouver la longueur de chacune des trois pièces, et le prix du mètre de chaque étoffe. 
 \end{enumerate}
  
 	\end{document}
